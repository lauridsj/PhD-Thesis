%!TEX root = main.tex
\cleardoublepage % beginne auf neuer seite
\selectlanguage{english}
\chapter*{Abstract}

Two key measurements and two phenomenological studies of top quark pair (\ttbar) production with the CMS experiment at the CERN Large Hadron Collider are presented. They include the first measurement of the \ttbar cross section at the new energy frontier, a detailed analysis of the \ttbar threshold culminating in the first observation of \ttbar bound state effects, and searches for new physics in \ttbar final states, such as heavy scalars, pseudoscalars or Axion-Like Particles.

The inclusive \ttbar production cross section \sigmatt is measured for the first time at the world-record energy of \sqrtsRIII, using \lumiRIII of early LHC Run~3 data. By combining the dilepton and lepton+jets (\ljets) decay channels and constraining the lepton and b tagging efficiencies \textit{in situ}, a precision of $3.4\%$, comparable to previous \sigmatt measurements, is achieved. This constitutes the first measurement of proton-proton scattering at \sqrtsRIII worldwide.

Furthermore, a search for heavy spin-0 states decaying to \ttbar using \lumiRII of LHC Run~2 data at \sqrtsRII in the dilepton channels is presented. The invariant \ttbar mass (\mtt) is combined with spin correlation observables to gain sensitivity to the spin and \CP structure of possible new intermediate states. 
The analysis is supported by a detailed modeling study of off-shell \ttbar production and of the interference between \ttbar and tW production.

An excess of events over the \ttbar continuum background is observed at low values of \mtt, with spin correlations consistent with a pseudoscalar state. It is interpreted in terms of a pseudoscalar \ttbar bound state \etat, and its cross section is measured to be $\sigetat =  8.7 \pm 1.1  \, \si{\pb}$ using a simplified model inspired by non-relativistic quantum chromodynamics. 
The excess is statistically significant at more than five standard deviations, constituting the first observation of \ttbar bound state effects.

Other interpretations of the observed excess are similarly possible. In particular, scenarios with generic pseudoscalar or scalar bosons are explored, and exclusion regions on their coupling to the top quark are derived both for the dilepton channels alone as well as in a combination with a separate analysis of the \ljets channels. In addition, Axion-Like Particles (ALPs) decaying to \ttbar are considered in the case of vanishing tree-level ALP-gluon couplings, while the more generic case is investigated phenomenologically in simulation.

\cleardoublepage % beginne auf neuer seite
\selectlanguage{ngerman}
\chapter*{Zusammenfassung}

\enlargethispage{2\baselineskip}

Es werden zwei essentielle Messungen und zwei ph{\"a}nomenologische Studien zur Produktion von Top-Quark-Paaren (\ttbar) mit dem CMS-Experiment am CERN Large Hadron Collider vorgestellt. 
Sie umfassen die erste Messung des \ttbar-Wir\-kungs\-quer\-schnitts bei der weltweit h{\"o}chsten Schwerpunktsenergie, eine detaillierte Analyse der \ttbar-Produktionsschwelle, die in die erste Beobachtung von einem gebundenen Zustand des \ttbar-Systems m{\"u}ndet, sowie Suchen nach neuer Physik in \ttbar-Endzust{\"a}nden, wie etwa schwere Skalar- oder Pseudoskalarbosonen oder Axion-Like Particles.

Der inklusive \ttbar-Produktionsquerschnitt \sigmatt wird zum ersten Mal bei \sqrtsRIII gemessen, unter Verwendung von fr{\"u}hen LHC Run~3-Daten mit integrierter Luminosit{\"a}t von \lumiRIII. Durch Kombination der Dilepton- und Lepton+Jets (\ljets)-Zerfallskan{\"a}le von \ttbar und simultane Bestimmung der Lepton- und b-tagging-Effizienzen \textit{in situ} wird eine Pr{\"a}zision von $3.4\%$ erreicht, die mit fr{\"u}heren \sigmatt-Messungen vergleichbar ist. 
Dies stellt die weltweit erste Messung von Proton-Proton-Streuprozessen bei \sqrtsRIII dar.

%Anschließend wird Produktion von off-shell \ttbar und \tttW-Interferenz mit dem Monte Carlo (MC)-Generator \bbfourl untersucht, der zum ersten Mal in der CMS-Simulation validiert und mit anderen MC-Generatoren verglichen wird und dabei zu einer deutlich verbesserten Beschreibung der Daten f{\"u}hrt.

Dar{\"u}ber hinaus wird eine Suche nach schweren Spin-0-Zust{\"a}nden, die zu \ttbar zerfallen, in den Dilepton-Kan{\"a}len mit \lumiRII Daten von LHC Run~2 bei \sqrtsRII vorgestellt. Die invariante Masse von \ttbar (\mtt) wird mit Spinkorrelations-Observablen kombiniert, um die Sensitivit{\"a}t gegen{\"u}ber dem Spin und der \CP-Struktur m{\"o}glicher neuer intermedi{\"a}rer Zust{\"a}nde zu erh{\"o}hen. 
Die Analyse wird durch detaillierte Studien zur Modellierung der Off-Shell-\ttbar-Produktion sowie zur Interferenz zwischen \ttbar- und tW-Produktion untermauert.

Ein {\"U}berschuss von Ereignissen gegen{\"u}ber dem \ttbar-Kontinuums-Hintergrund wird bei niedrigen Werten von \mtt und mit Spinkorrelationen konsistent mit einem pseudoskalaren Zustand beobachtet. Er wird als pseudoskalarer gebundener \ttbar-Zustand \etat interpretiert, und dessen Produktionsquerschnitt wird mithilfe eines vereinfachten, von nichtrelativistischer Quantenchromodynamik inspirierten Modells zu $\sigetat =  8.7 \pm 1.1  \, \si{\pb}$ gemessen. 
Der {\"U}berschuss ist mit einer Signifikanz von mehr als f{\"u}nf Standardabweichungen statistisch belegt und stellt somit die erste Beobachtung von gebundenen Zust{\"a}nden im \ttbar-System dar.

Weitere Interpretationen des beobachteten Überschusses sind ebenfalls m{\"o}glich.
Insbesondere werden Szenarios mit generischen pseudoskalaren oder skalaren Bosonen untersucht, und Ausschlussregionen 
hinsichtlich ihrer Kopplungen an das Top-Quark werden sowohl f{\"u}r die Dilepton-Kan{\"a}le allein als in Kombination mit einer separaten Analyse der \ljets-Kan{\"a}le berechnet. Zus{\"a}tzlich werden zu \ttbar zerfallende Axion-Like Particles (ALPs) betrachtet: einerseits im Fall verschwindender ALP-Gluon-Kopplungen, andererseits im allgemeinen Fall, der ph{\"a}nomenologisch in Simulationen untersucht wird.

\selectlanguage{english}