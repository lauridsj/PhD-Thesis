%!TEX root = main.tex
\cleardoublepage % beginne auf neuer seite
\selectlanguage{english}
\chapter*{Abstract}

Two measurements and two phenomenological studies of top quark pair (\ttbar) production at the CMS experiment are presented. The inclusive \ttbar production cross section \sigmatt is measured for the first time at \sqrtsRIII, using \lumiRIII of early LHC Run~3 data. By combining the dilepton and lepton+jets (\ljets) decay channels of \ttbar and constraining the lepton and b tagging efficiencies \textit{in situ}, a precision of $3.4\%$ comparable to previous \sigmatt measurements is achieved.

Following this, a study of off-shell \ttbar production and \tttW interference is performed using the Monte Carlo (MC) generator \bbfourl, which is validated for the first time in CMS simulation and compared to other MC generators, finding a significantly improved description of the data.

Finally, a search for spin-0 states decaying to \ttbar in the dilepton channels is presented, using \lumiRII of LHC Run~2 data at \sqrtsRII. The invariant \ttbar mass (\mtt) is combined with spin correlation observables to gain sensitivity to the spin and \CP structure of possible new intermediate states. A statistically significant excess of events over the \ttbar continuum background is observed at low values of \mtt and for spin correlations consistent with a pseudoscalar state. It is interpreted in terms of a pseudoscalar \ttbar bound state \etat, and its cross section is measured to be $\sigetat =  8.7 \pm 1.1  \, \si{\pb}$ using a simplified model inspired by non-relativistic quantum chromodynamics. 

The same data is further interpreted in terms of generic pseudoscalar or scalar bosons, and exclusion regions on their coupling to the top quark are derived both for the dilepton channels alone as well as in a combination with a separate analysis of the \ljets channels. As a third interpretation of the data, Axion-Like Particles (ALPs) decaying to \ttbar are considered in the case of vanishing tree-level ALP-gluon couplings, while the more generic case is studied phenomenologically in simulation.

\cleardoublepage % beginne auf neuer seite
\selectlanguage{ngerman}
\chapter*{Zusammenfassung}

Es werden zwei Messungen und zwei ph{\"a}nomenologische Studien zur Produktion von Top-Quark-Paaren (\ttbar) am CMS-Experiment vorgestellt. Der inklusive \ttbar-Produktionsquerschnitt \sigmatt wird zum ersten Mal bei \sqrtsRIII gemessen, unter Verwendung von fr{\"u}hen LHC Run~3-Daten mit integrierter Luminosit{\"a}t von \lumiRIII. Durch Kombination der Dilepton- und Lepton+Jets (\ljets)-Zerfallskan{\"a}le von \ttbar und simultane Bestimmung der Lepton- und b-tagging-Effizienzen \textit{in situ} wird eine mit fr{\"u}heren \sigmatt-Messungen vergleichbare Pr{\"a}zision von $3.4\%$ erreicht.

Anschließend wird Produktion von off-shell \ttbar und \tttW-Interferenz mit dem Monte Carlo (MC)-Generator \bbfourl untersucht, der zum ersten Mal in der CMS-Simulation validiert und mit anderen MC-Generatoren verglichen wird und dabei zu einer deutlich verbesserten Beschreibung der Daten f{\"u}hrt.

Schlie{\ss}lich wird eine Suche f{\"u}r nach \ttbar zerfallende Spin-0-Zust{\"a}nde in den Dilepton-Kan{\"a}len mit Daten von LHC Run~2, einer integrierten Luminosit{\"a}t von \lumiRII und \sqrtsRII vorgestellt. Die invariante Masse von \ttbar (\mtt) wird mit Spinkorrelations-Observablen kombiniert, um die Sensitivit{\"a}t gegen{\"u}ber dem Spin und der \CP-Struktur m{\"o}glicher neuer intermedi{\"a}rer Zust{\"a}nde zu erh{\"o}hen. Ein statistisch signifikanter {\"U}berschuss von Ereignissen im Vergleich zum \ttbar-Kontinuums-Hintergrund wird bei geringen Werten von \mtt und f{\"u}r Spinkorrelationen konsistent mit einem pseudoskalaren Zustand beobachtet. Der {\"U}berschuss wird als pseudoskalarer gebundener \ttbar-Zustand \etat interpretiert, und dessen Produktionsquerschnitt wird mithilfe eines vereinfachten, von nichtrelativistischer Quantenchromodynamik inspirierten Modells zu $\sigetat =  8.7 \pm 1.1  \, \si{\pb}$ gemessen.

Die selben Daten werden weiterhin als generische pseudoskalare oder skalare Bosonen interpretiert, und Ausschlussregionen 
hinsichtlich ihrer Kopplungen zum Top-Quark werden sowohl f{\"u}r die Dilepton-Kan{\"a}le allein als auch f{\"u}r eine Kombination mit einer separaten Analyse der \ljets-Kan{\"a}le berechnet. Als eine dritte Interpretation der Daten werden zu \ttbar zerfallende Axion-Like Particles (ALPs) im Fall verschwindender ALP-Gluon-Kopplungen betrachtet, w{\"a}hrend der allgemeinere Fall auf ph{\"a}nomenologische Weise in Simulationsdaten untersucht wird.

\selectlanguage{english}