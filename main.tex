\documentclass[12pt]{report}

\usepackage[utf8]{inputenc}
\usepackage{amsmath, amssymb, amsthm}
\usepackage{siunitx}
\usepackage{xspace}

\DeclareSIUnit{\GeV}{\giga\electronvolt}
\DeclareSIUnit{\TeV}{\tera\electronvolt}

\newcommand{\ttbar}{\ensuremath{\mathrm{t \bar{t}}}\xspace}

\newcommand{\pt}{\ensuremath{p_T}\xspace}
\newcommand{\abseta}{\ensuremath{|\eta|}\xspace}
\newcommand{\sigmatt}{\ensuremath{\sigma_{\ttbar}}\xspace}
\newcommand{\mll}{\ensuremath{m_{\ell\ell}}\xspace}
\newcommand{\Rinout}{\ensuremath{R_{\mathrm{in/out}}}\xspace}


\newcommand{\ljets}{\ensuremath{\ell\mathrm{+jets}}\xspace}
\newcommand{\emu}{\ensuremath{\mathrm{e\mu}}\xspace}

\newcommand{\sqrtsRIII}{$\sqrt{s} = \SI{13.6}{\TeV}$}
\newcommand{\xsecpred}{\ensuremath{921\hspace{1pt}^{+29}_{-37}\unit{pb}}\xspace}

\title{Measurement of the inclusive \ttbar production cross section and searches for new physics in \ttbar final states}
\author{Laurids Jeppe}
\date{January 2023}

\begin{document}

\maketitle

\newpage
\tableofcontents
\newpage

\chapter{Introduction}

\chapter{Theoretical Basics}
\label{ch:theory}

\chapter{The CMS detector}
\label{ch:detector}

\chapter{Data reconstruction}
\label{ch:reco}

\chapter{Measurement of the inclusive \ttbartitle cross section at \texorpdfstring{$\sqrt{s}$}{sqrt(s)} = 13.6 TeV}
\label{ch:ttxs}

%This chapter describes the measurement of the inclusive \ttbar production cross section at the LHC at the center-of-mass energy of \sqrtsRIII. First, a motivation and overview of the analysis design will be given. Then, object and cut definitions as well as several applied corrections will be explained in detail. Finally, the likelihood fit used to extract the cross section, including all uncertainties, is described and the result discussed.

\section{Introduction}

% motivation: new com energy, possible glimpse at new physics; new run, new calibrations, top physics requires most objects -> good check of performance

% method needs to be tuned for early measurement
% special mention: lepton sf -> can be estimated in situ
% do one fit with and one without

In July 2022, the LHC officially resumed collecting data after over three years of shutdown, thereby starting LHC Run~3. It did so at a new, unprecedented center-of-mass energy of \sqrtsRIII, inviting the experiments to measure physical observables at the new energy frontier.

One important such observable is the inclusive \ttbar production cross section. It is, in essence, the total rate of top quark pair production at the LHC, integrated over the kinematic distributions of the particles produced. As mentioned in \cref{ch:theory}, the top quark has a special place in the standard model as the heaviest known elementary particle, as well as the only colored particle that decays before hadronizing. 
It is thus important for many potential BSM scenarios, such as models with additional Higgs bosons, which might couple strongly to the top quark. 
As such, measurements of top quark-related observables at the highest possible energies are attractive tests of the SM. The inclusive \ttbar production cross section, as one of the simplest top quark observables, is well suited for a first measurement at the new center-of-mass energy.

Simultaneously, restarting such a large experiment as CMS after a three-year shutdown poses many experimental challenges. Due to the change in energy, as well as physical changes in the accelerator and detector, new calibrations as well as validations of some previous calibrations are required to ensure that the detector performance is understood. An early measurement of the inclusive \ttbar cross section is well suited to serve as such a cross-check: Because of the decay chain of the top quark, a top quark measurement involves many of the different objects reconstructed at CMS, which
%, as described in Ch. \ref{sec:methods:reco}, 
allows for a check of a wide landscape of calibrations.

The measurement described in this chapter was designed specifically with these motivations in mind, and as such exhibits several novel features. Firstly, it combines events from both the dilepton and \ljets decay channels of \ttbar, categorized by lepton flavor content, combining the higher statistics of the \ljets channel with the high purity of the \emu channel and allowing to constrain uncertainties on the lepton identification efficiency directly from the data. This is done using a simultaneous maximum likelihood fit to the event yields in the different categories, with experimental and theoretical uncertainties treated as nuisance parameters.

Secondly, the events are additionally categorized by their number of b-tagged jets, which similarly allows for an in-situ measurement of the b-tagging efficiencies. This averts needing to wait for external b-tagging calibrations, allowing for a measurement as early as possible.

The results of this work were first presented as a Physics Analysis Summary in September~2022~\cite{CMS:TOP-22-012-PAS}, only two months after the start of data taking, as the first public physics result of LHC Run~3. It was later published in \textit{JHEP} as \citere{CMS:TOP-22-012}, again representing the first published Run~3 result. A similar result by ATLAS was later published in \citere{ATLAS:2023slx}.

This chapter is structured as follows: In \cref{sec:ttxs:setup}, the used data sets, object definitions, and event selection criteria are described, followed by the derivation and application of needed corrections in \cref{sec:ttxs:corrections}, and the resulting data-MC agreement is shown in \cref{sec:ttxs:control}. The considered systematic uncertainties are listed in \cref{sec:ttxs:systematics}, and the fit results are presented in \cref{sec:ttxs:fitresults}. The chapter is concluded by a short summary and outlook in \cref{sec:ttxs:summary}.

\section{Data sets and event selection}
\label{sec:ttxs:setup}

In this section, the choice of data sets for experimental data and for simulation, as well as the choice of triggers, is described. Following that, the object and event selection procedure is outlined and several event categories to be used in the likelihood fit are defined. 

\subsection{Data sets}
\label{sec:ttxs:datasets}

\paragraph{Experimental data}
The measurement is performed on data recorded during the period between July 27\textsuperscript{th} and August 02\textsuperscript{nd} 2022, corresponding to an integrated luminosity of \lumiRIII. This amount of data is chosen as a balance between sensitivity and speed for the early measurement: It roughly corresponds to the point where the measurement precision is no longer primarily limited by the quantity of the data, while at the same time restricting to a data set where beam and detector conditions were stable and comparable to the data-taking in Run~2.

% TODO luminosity measurement

Both single-lepton and dilepton triggers were used to select events used in this measurement during detector operation, identifying leptons in the range of $\abseta < 2.5$. The \pt requirements of the triggers are summarized in Tab.~\ref{tab:ttxs:triggers}.

\begin{table}
    \centering
    \begin{tabular}{c|c}
        Trigger & Lepton requirement \\
        \hline
        \hline
        \ejets & e($\pt > 32$ GeV) \\
        \mujets & \textmu($\pt > 27$ GeV) \\
        \ee & e($\pt > 23$ GeV) and e($\pt > 12$ GeV) \\
        \mumu & \textmu($\pt > 17$ GeV) and \textmu($\pt > 8$ GeV) \\
        \emu & e($\pt > 23$ GeV) and \textmu($\pt > 8$ GeV)  or \\
        & e($\pt > 12$ GeV) and \textmu($\pt > 23$ GeV)
    \end{tabular}
    \caption{\textbf{Trigger definitions} as used for the \ttbar cross section measurement. The leptons are required to be isolated and in the pseudorapidity range  $\abseta < 2.5$.}
    \label{tab:ttxs:triggers}
\end{table}

\paragraph{Simulation}
To compare the data with predictions, Monte Carlo (MC) simulation is used to simulate both the \ttbar signal as well as most important background processes, specifically single-top quark production in the $t$-channel, associated tW production, Z+jets production, W+jets production, and diboson (WW, WZ and ZZ) production. The MC generator \textsc{Powheg v2}~\cite{Powheg:2004, Powheg:2007, Powheg:2010} is used to generate \ttbar, $t$-channel single-top, and tW events at next-to-leading order (NLO) in perturbative QCD, while the generators \textsc{MadGraph5\_aMC@NLO}~\cite{MG5aMCatNLO:2014} and \textsc{Pythia 8}~\cite{Pythia:2015} are used to generate Z+jets/W+jets and diboson events, respectively, at leading order (LO). For Z+jets and W+jets, up to four additional jets are included in the matrix element using the MLM matching scheme~\cite{Mangano:2006rw}. For $t$-channel single-top, \textsc{MadSpin} is used to simulate the top decay.

All of the generated events are interfaced to \textsc{Pythia 8} for parton showering and hadronization, and further processed in a full simulation of the CMS detector as described in \cref{ch:mc}. The proton structure in the matrix element calculation is described by the NNPDF3.1 parton distribution function (PDF) set at NNLO. Note that another background contribution, from QCD-produced multijet events with fake or non-prompt leptons, is not simulated, but estimated from data (see \cref{sec:ttxs:datadriven}).

Theoretical predictions, as well as the measured integrated luminosity, are used to normalize the cross sections of the signal and background samples as follows: The \ttbar signal, is normalized to a cross section of \xsecpred computed at NNLO+NNLL in QCD~\cite{Czakon:2011xx}, which is also used as a prediction for comparison with the SM. For the other backgrounds, the following orders in QCD and methods or programs are used: \textsc{MCFM}~\cite{Campbell:2020fhf} (NNLO) for single-top, \textsc{DYTurbo}~\cite{Camarda:2019zyx} (NNLO) for W+jets and Z+jets, \textsc{Matrix}~\cite{Grazzini:2017mhc} (NLO) for diboson, and an NNLO calculation from Ref. \cite{Kidonakis:2021vob} for tW.


\subsection{Object definition}
\label{sec:ttxs:objects}

\paragraph{Leptons}

Electrons or muons are considered for the analysis if they have $\pt > 10$ GeV and $\abseta < 2.4$. For electrons, the range $1.44 < \abseta < 1.57$, corresponding to the transition region between barrel and endcaps in the ECAL, is removed. Furthermore, additional identification criteria (ID) are applied to remove non-prompt or fake (i.e. wrongly reconstructed) leptons and enrich the selection with \ttbar events.

For electrons, the ``tight'' working point of the cut-based ID described in \citere{CMS:EGM-17-001} is applied, which includes information from both the details of the electromagnetic shower in the ECAL and the track, as well as the matching between the two. It also includes a requirement for the electron to be isolated from other particles such as hadrons, which is implemented in the form of the relative isolation variable \Irel. It is defined as the scalar \pt sum of all particles in a cone around the lepton in question, divided by the lepton \pt. Here, $\Delta R = \sqrt{(\Delta \eta)^2 + (\Delta \varphi)^2} < 0.3$ is used for the radius of the cone. Additional corrections accounting for pileup particles are applied.

For muons, a similar cut-based ID is used as described in \citere{CMS:MUO-16-001}, also at the tight working point. Here, criteria on the compatibility of tracks in the inner tracker, the muon detectors and the reconstructed primary vertex are employed. Again, a cut on \Irel is used, defined equivalently but with a cone size of $\Delta R < 0.4$.

\paragraph{Jets}

The anti-$k_T$ algorithm~\cite{Cacciari:2008gp} is used to cluster reconstructed particles into jets with a distance parameter of 0.4. In order for a jet to be considered, it is required to have $\pt > 30$ GeV and $\abseta < 2.4$, and jets overlapping with any considered leptons (i.e. fulfilling the above criteria) are removed. %Again, additional ID criteria, based on the fractions of \pt originating from neutral hadron, charged hadron and electromagnetic jet constituents, are used to reject 

\paragraph{Tagging of b jets}

A special role is played by jets originating from the showering and hadronization of b quarks. Naively, two such jets are expected per \ttbar event from the two top decays, although in practice one or both jets may fall out of acceptance of the detector or otherwise not be identified. Furthermore, additional b quarks may be produced by radiation at higher orders in QCD. Correctly tagging these jets as such can greatly improve signal purity by cutting away backgrounds such as Z+jets, W+jets and QCD multijet events.

Here, the \textsc{DeepJet} algorithm~\cite{DeepJet:2020,CMS:BTV-16-002}, which is based on a deep neural network (DNN) classifier, is used to identify (``tag'') b jets. A working point with an identification efficiency of more then 75\% is used, with misidentification rates of around 17\% for charm jets and around 1\% for other jets from light quarks or gluons.

\subsection{Channel definition}
\label{sec:ttxs:channels}

Events are selected with either one or two leptons, corresponding respectively to the \ljets and dilepton decay channels of \ttbar. They are categorized into separate channels by their lepton flavor content, and additional requirements are applied for the different channels. 

\paragraph{Dilepton channels}

Events with exactly two leptons, required to have opposite electric charge, are sorted into three dilepton channels (\emu, \ee, and \mumu). The presence of at least one jet is required, and in the same-flavor channels (\ee and \mumu), at least one jet is required to be b tagged in order to reject Z+jets and QCD multijet background. In the much purer \emu channel, on the other hand, events without b tags are retained to later help constrain the b tagging efficiency in the fit to data.

In order to reject even more Z+jets background, events in the same-flavor channels with an invariant dilepton mass of $| \mll - m_Z | < 15$ GeV, where $m_Z$ is the Z boson mass, are removed.

\paragraph{\ljets channels}

Events with exactly one lepton are sorted into the \ejets or \mujets channels based on their flavor. At least three jets are required, of which at least one needs to be b tagged. Note that regardless of these selections, there is still non-negligible background from QCD multijet events where the lepton is non-prompt or fake, which is estimated from data (see \cref{sec:ttxs:datadriven}).

\paragraph{\pt requirement}

In all channels, the considered leptons are required to have $\pt > 35$ GeV. This requirement is needed in the \ljets channels in order to stay above the single-lepton trigger \pt thresholds (compare Tab. \ref{tab:ttxs:triggers}). In this measurement, the choice is made to apply the same \pt requirement also both leptons in the dilepton channels to ensure consistency between the lepton definitions. This is done to help constrain the lepton ID scale factors using the combination of lepton flavor channels, which otherwise might not be accurate since the scale factors for different lepton definitions might differ. In particular it opens up the possibility to extract a result on the cross section without any prior knowledge on the lepton ID efficiencies, which was done in the first published version of this analysis \cite{CMS:TOP-22-012-PAS}. %\todo{decide on whether to add this cross check here}% and is included in this thesis as a cross-check (see sec. ??).

\paragraph{b tag and jet categorization}

In practice, the efficiency of the b tagging algorithm used might be different between simulation and data, necessitating a correction to prevent bias. In this analysis, this efficiency is measured simultaneously with the cross section directly in the data. To do so, the lepton flavor channels are additionally split into categories based on the number (exactly 0, 1, or 2) of b tagged jets. Since only the \emu channel allows events with 0 b tags, this results in 11 categories total. To gain further sensitivity to the b tagging efficiency and to increase possible separation between \ttbar signal and background, the selected events are finally coarsely binned into the number of accepted jets for the eventual fit, giving a total number of 40 bins.


\section{Corrections}
\label{sec:ttxs:corrections}

While the simulation used in CMS tries to describe as many physics and detector effects as possible, in practice it should always be expected that not all observables agree with the experimental data perfectly. This is especially true for an early analysis such as this, as the detector conditions might have changed significantly during the long shutdown between LHC Runs 2 and 3, and the simulation had not been recalibrated at the time of the measurement. 

Because of this, the analysis setup is designed to either directly measure or cross-check as many required experimental calibration and correction factors as possible. This includes pileup corrections, efficiency scale factors for triggers, electrons, muons and b tags, as well as jet energy corrections, all of which are briefly described in this section.

In addition to these experimental corrections, background processes might also be imperfectly described by the simulation because of theoretical shortcomings. In this case, ways have to be found to correct them directly from the experimental data. Here, two such cases are relevant and will be presented in the latter half of this section: The Z+jets background in the dilepton channels and in the presence of b tagged jets, for which the normalization is taken from data; and the QCD background in the \ljets channels, which uses a fully data-driven estimation and foregoes simulation entirely.

\subsection{Experimental corrections}
\label{sec:ttxs:scalefactors}

\paragraph{Pileup reweighting}

The simulation samples used in this analysis were generated before the start of Run 3 data taking using a projected estimate of the average pileup. As a result, the pileup distribution in the simulation does not match the one observed in data, which could influence mostly jet-related variables such as the number of jets and the jet \pt.

Since at the time of the measurement, no theory-based calculation for the correct pileup distribution were available, an experimental approach was taken. Three experimental observables that are strongly correlated with pileup were identified: 

\begin{itemize}
    \item The number of well-reconstructed primary vertices per event $n_{\mathrm{PV}}$;
    \item The median \pt density in the calorimeter, calculated from calorimeter-only jets as $\rho^{\mathrm{calo}} = \mathrm{med} (\pt / A)$, where $A$ is the jet area defined in the $\varphi$-$\eta$ plane and the median is taken over all jets in the event;
    \item The median \pt density in the tracker $\rho^{\mathrm{trk}}$, defined equivalently as $\rho^{\mathrm{calo}}$, but for jets calculated only from tracker information.
\end{itemize}

A binned reweighting from simulation to data is derived for each observable based on the full data sample, and the average of the three weights is applied to the simulation, so that approximate agreement is achieved in all three variables. The distributions before and after reweighting can be seen in \cref{fig:ttxs:pileup}.

% outputs/2022_datamc_run3jec_nopileup_300822
% outputs/2022_datamc_ul18jec_pu_triggersf_310822

\begin{figure}[p]
    \centering
    \includegraphics[width=0.49 \textwidth]{figures/ttxs/pileup/nvtx_orig.pdf}
    \hfill
    \includegraphics[width=0.49 \textwidth]{figures/ttxs/pileup/nvtx_reweighted.pdf}
    \includegraphics[width=0.49 \textwidth]{figures/ttxs/pileup/rhoFastjetCentralChargedPileUp_orig.pdf}
    \hfill
    \includegraphics[width=0.49 \textwidth]{figures/ttxs/pileup/rhoFastjetCentralChargedPileUp_reweighted.pdf}
    \includegraphics[width=0.49 \textwidth]{figures/ttxs/pileup/rhoFastjetCentralCalo_orig.pdf}
    \hfill
    \includegraphics[width=0.49 \textwidth]{figures/ttxs/pileup/rhoFastjetCentralCalo_reweighted.pdf}
    \caption{\textbf{Pileup reweighting.} Pileup-related distributions in MC and data in before (left) and after reweighting (right). From top to bottom: number of primary vertices as well as the mean energy densities $\rho^{\mathrm{trk}}$ (calculated using tracker input) and $\rho^{\mathrm{calo}}$ (calculated using calorimeter input).}
    \label{fig:ttxs:pileup}
  \end{figure}

\paragraph{Trigger scale factors}

The trigger efficiency, i.e. the probability for an event falling into the selection phase space to be triggered by the low- and high-level triggers, can differ between simulation and data.
In principle, both dilepton and single-lepton triggers are used for this measurement and should be considered for the efficiency calculation. However, due to the high offline \pt requirements for the two leptons applied in all channels, the fraction of events that are triggered only by the dilepton triggers is negligibly small, and can be neglected for the purpose of determining the scale factor. Thus, only the single-lepton triggers are considered in this section for simplicity.
%It should be noted that, while both dilepton and single-lepton triggers are used for the measurement, the single-lepton triggers by far dominate the selected event count due to the high \pt thresholds required, and so the efficiency is measured for the single-lepton triggers only for simplicity.

The efficiency measurement is performed by the so-called tag-and-probe (T\&P) method, using $\mathrm{Z} \rightarrow \mathrm{e^+ e^-}$ and $\mathrm{Z} \rightarrow \mathrm{\mu^+ \mu^-}$ events. They are selected using the same definitions presented above, including the lepton identification, except for requiring their invariant mass to fulfill $| \mll - m_Z | < 20$ GeV. At least one of the leptons is required to pass the relevant single-lepton trigger and is then designated the tag, while the other lepton might or might not pass the trigger and is designated the probe. Assuming the probability for the two leptons to pass the trigger to be independent of each other, the trigger efficiency, given by probability of the probe to pass, can be written as

\begin{equation}
    \epsilon_{\mathrm{tr}} = \frac{N (\text{Probe passes})}{ N (\text{Probe passes}) + \frac{1}{2} N (\text{Probe fails}) }
\end{equation}

where $N$ corresponds to to the number of events in which the second lepton either passes or fails the trigger, and the combinatoric factor $\frac{1}{2}$ comes from the fact that either one or the other lepton can fail. 

The efficiency is measured in this way in coarse bins of lepton \pt and \abseta, separately for muons and electrons, in both simulation and experimental data. It is then applied to simulation in the following way: For \ljets events, a simple ratio $\epsilon_{\mathrm{tr,data}} / \epsilon_{\mathrm{tr,sim}}$ is applied to each simulation event as a scale factor, which is displayed in \cref{fig:ttxs:triggersf}. For dilepton events, on the other hand, the fact that only one lepton needs to pass the single-lepton trigger needs to be taken into account. This leads to a per-event efficiency given by

\begin{equation}
\label{eq:ttxs:triggersf}
    \epsilon_{\mathrm{tr,\ell \ell}} = \epsilon_{\mathrm{tr,\ell 1}} + \epsilon_{\mathrm{tr,\ell 2}} - \epsilon_{\mathrm{tr,\ell 1}} \epsilon_{\mathrm{tr,\ell 2}}
\end{equation}

where $\epsilon_{\mathrm{tr,\ell 1}}$ and $\epsilon_{\mathrm{tr,\ell 2}}$ are the efficiencies evaluated at the \pt and \abseta of the two leptons, respectively. Again, the ratio of this event efficiency in data and simulation is applied to the simulation.

\begin{figure}[t]
    \centering
    \includegraphics[width=0.49 \textwidth]{figures/ttxs/scalefactors/trigger_eff_e.pdf}
    \hfill
    \includegraphics[width=0.49 \textwidth]{figures/ttxs/scalefactors/trigger_eff_mu.pdf}
    \caption{\textbf{Trigger scale factors.} Single-lepton trigger efficiencies in data and MC (top) and scale factors (bottom) for electrons (left) and muons (right) as a function of lepton \pt and $|\eta|$, calculated using a tag-and-probe-method. The error bars designate both statistical and systematic uncertainties.}
    \label{fig:ttxs:triggersf}
\end{figure}

\paragraph{Lepton scale factors}

Similarly to the triggers, the reconstruction and identification of leptons can exhibit different efficiencies between simulation and data, and thus require scale factors. 
%Two orthogonal approaches are taken with respect to this: In the first approach, 
The efficiencies are measured with a similar tag-and-probe method as for the triggers, and the simulation is corrected to the data. This is the standard approach commonly taken in CMS, detailed in \citeres{CMS:EGM-17-001,CMS:MUO-16-001} for electrons and muons, respectively.%, and will be used in the figures presented in this thesis unless stated otherwise. 
The efficiency measurement was not performed as part of this thesis, but is still shown in \cref{fig:ttxs:muonsf,fig:ttxs:elesf} for reference. The muon scale factors are split into a reconstruction and an identification part, while these are combined for the electron scale factors.  

\begin{figure}[t]
    \centering
    \includegraphics[width=0.49 \textwidth]{figures/ttxs/scalefactors/lepton_eff_mu_reco.pdf}
    \hfill
    \includegraphics[width=0.49 \textwidth]{figures/ttxs/scalefactors/lepton_eff_mu_id.pdf}
    \caption{\textbf{Muon scale factors}. Muon efficiencies in data and MC (top) and scale factors (bottom), split into reconstruction (left) and identification (right) and shown as a function of lepton \pt and $|\eta|$, calculated using a tag-and-probe-method. The error bars designate both statistical and systematic uncertainties.}
    \label{fig:ttxs:muonsf}
  \end{figure}
  
  \begin{figure}[t]
    \centering
    \includegraphics[width=0.49 \textwidth]{figures/ttxs/scalefactors/lepton_eff_e.pdf}
    \caption{\textbf{Electron scale factors.} Combined electron efficiencies in data and MC (top) and scale factors (bottom) as a function of lepton \pt and $|\eta|$, calculated using a tag-and-probe-method. The error bars designate both statistical and systematic uncertainties.}
    \label{fig:ttxs:elesf}
  \end{figure}

%In the second, alternative, approach, on the other hand, no correction to the simulation is made, and instead the lepton efficiencies in data in the selection phase space are measured simultaneously with the cross section using the likelihood fit. This will be further described in sec. ??.

\paragraph{b tagging scale factors}

The performance of b tagging algorithms, such as the \textsc{DeepJet} algorithm used in this analysis, is known to differ between simulation and data, necessitating corrections. This is particularly relevant here, as the multivariate classifier behind \textsc{DeepJet} had not been re-trained on Run~3 data at the time of the measurement; instead, the Run~2 calibration was used. 

Since no external calibration of b tagging efficiencies for Run~3 was available within the timeframe of this study, the b tagging efficiency in data is extracted directly from the data itself. This is achieved by performing a simultaneous likelihood fit with the ttbar cross section, as described in \cref{sec:ttxs:systematics}. As a result, no b tagging scale factors are applied beforehand. 

\paragraph{Jet energy corrections}

Another observable that often differs significantly between observed data and simulation is the measured energy response of the jets. Both its mean value, the jet energy scale (JES), and the jet energy resolution (JER) require corrections, which are together referred as jet energy corrections (JECs). Both are centrally provided by CMS following the methods of \citere{CMS:JME-13-004}. 

The derivation of the JES is performed in multiple steps:
first, the expected fraction of jet energy due to pileup, determined from MC simulation, is removed from all jets in data and MC. Second, the difference in jet energy between detector- and particle-level jets in simulation is determined as a function of jet kinematics, and detector-level jets are corrected accordingly in both data and simulation.
Third, residual disagreements of the simulation with the data are corrected using experimental jet measurements in dijet, $\gamma$+jets, Z+jets, and multijet events, again parametrized as a function of jet kinematics~\cite{CMS:JME-13-004}.

Similarly, JER scale factors are determined by correcting the resolution in simulation to the one seen in data, based on dijet, $\gamma$+jets and Z+jets events~\cite{CMS:JME-10-011}. They are then applied to jets in simulation by scaling the difference between detector- and particle-level jet energy for jets where a matched particle-level jet is found, while a stochastic smearing is used otherwise.


\subsection{Data-driven background estimation}
\label{sec:ttxs:datadriven}

\paragraph{QCD background}

A significant background contribution in the \ljets channels, especially in the categories with only one b tag, is given by QCD multijet events with one reconstructed lepton. The lepton in question might be non-prompt, e.g. from radiated photons splitting into leptons or decays of heavy flavor hadrons, or it might be fake, i.e. a different particle (such as a photon or pion in the case of electrons) misidentified as a lepton.

It is often not practical to estimate this background using MC simulation as is done for the other backgrounds in this analysis. The reason is that, due to the large cross section of QCD multijet events at the LHC but low ratio of events with a fake or non-prompt lepton, very large MC data sets are needed to achieve significant statistics in the selected phase space, requiring excessive computing power. In addition to that, fake leptons are known not to be well-described by the simulation.

Instead, a fully data-driven approach is taken to estimate the QCD background in the \ljets channels. For this, multiple control regions (CRs) orthogonal to the signal region (SR) are defined. In the first CR, denoted ``QCD CR'', the same cuts as in the SR are applied, except that the requirement for the single lepton to be isolated from other particles (\Irel, see \cref{sec:ttxs:objects}) is inverted. It is expected that QCD events that fall in either the QCD CR or the SR show similar shapes in observable distributions, as long as said observables are uncorrelated with the lepton isolation. Thus, the shape of the QCD background can be extracted from the CR and applied in the SR. \cref{fig:ttxs:np_plots_mj_CR,fig:ttxs:np_plots_ej_CR} show the distributions of several key distributions for the QCD background in the \mujets and \ejets channels, respectively, which is estimated by subtracting all simulated (MC) processes from the data.

\begin{figure}[!ht]
    \centering
    \includegraphics[width=0.49 \textwidth]{figures/ttxs/np_plots_run3/lep_pt_mj_CR.pdf}
    \hfill
    \includegraphics[width=0.49 \textwidth]{figures/ttxs/np_plots_run3/lep_eta_mj_CR.pdf}
    \includegraphics[width=0.49 \textwidth]{figures/ttxs/np_plots_run3/njet_mj_CR.pdf}
    \hfill
    \includegraphics[width=0.49 \textwidth]{figures/ttxs/np_plots_run3/nbtag_mj_CR.pdf}
    \caption{\textbf{QCD control region for \mujets.} Distributions in the QCD CR for the \mujets channel. From top left to bottom right: $\pt$ of the lepton, $\eta$ of the lepton, the number of jets, and the number of b-tagged jets. The uncertainty bands include MC statistical and systematic uncertainties. The difference between data and MC prediction is considered QCD background and shown in green.}
    \label{fig:ttxs:np_plots_mj_CR}
\end{figure}

\begin{figure}[!ht]
    \centering
    \includegraphics[width=0.49 \textwidth]{figures/ttxs/np_plots_run3/lep_pt_ej_CR.pdf}
    \hfill
    \includegraphics[width=0.49 \textwidth]{figures/ttxs/np_plots_run3/lep_eta_ej_CR.pdf}
    \includegraphics[width=0.49 \textwidth]{figures/ttxs/np_plots_run3/njet_ej_CR.pdf}
    \hfill
    \includegraphics[width=0.49 \textwidth]{figures/ttxs/np_plots_run3/nbtag_ej_CR.pdf}
    \caption{\textbf{QCD control region for \ejets.} Distributions in the QCD CR for the \ejets channel, same as in \cref{fig:ttxs:np_plots_mj_CR}. The difference between data and MC prediction is considered QCD background and shown in green.}
    \label{fig:ttxs:np_plots_ej_CR}
  \end{figure}

The normalization of the QCD background is fixed through the so-called \textit{ABCD method}~\cite{CDF:1995eix,CDF:2000gwd}, for which an additional CR (the ``1-jet CR'') is defined.
It again contains events that pass the main selection, except for requiring exactly one jet (as opposed to at least three jets in the SR or QCD CR). These events are enriched with QCD events and contain negligible amounts of \ttbar signal. They are used to measure the ratio $f_{\mathrm{fake}}$ of QCD events that pass or fail the lepton isolation requirement, given by

\begin{equation}
\label{eq:ttxs:abcd_fakerate}
    f_{\mathrm{fake}} = \frac{ N_{\text{1 jet, pass}}^{\text{data}} - N_{\text{1 jet, pass}}^{\text{MC}} }{ N_{\text{1 jet, fail}}^{\text{data}} - N_{\text{1 jet, fail}}^{\text{MC}} }
\end{equation}

\noindent where $N_{\text{1 jet, pass}}$ and $N_{\text{1 jet, fail}}$ denote 1-jet-events that pass and fail the lepton isolation requirement, respectively; ``data'' refers to the experimental data, and ``MC'' refers to the sum of all non-QCD processes, which are estimated by MC simulation. Here, this ratio is measured in four coarse bins of lepton \pt and \abseta to accurately model lepton-related distributions; it can be seen in \cref{fig:ttxs:fakerate}.

\begin{figure}[!ht]
    \centering
    \includegraphics[width=0.58 \textwidth]{figures/ttxs/np_plots_run3/fakerate.pdf}
    \caption{\textbf{QCD fake rate}. The fake rate for the QCD background estimated in the 1 jet bin, separately for electrons and muons as, a function of lepton \pt and $|\eta|$. The error bars designate statistical uncertainties only.}
    \label{fig:ttxs:fakerate}
\end{figure}

Naively, the full distribution of the QCD background in the SR for any observable can then be written as

\begin{equation}
\label{eq:ttxs:abcd_naive}
    N_{\text{SR}}^{\text{QCD}} = ( N_{\text{CR}}^{\text{data}} - N_{\text{CR}}^{\text{MC}} ) \times f_{\mathrm{fake}}
\end{equation}

\noindent where $N_{\text{CR}}^{\text{data}}$ and $N_{\text{CR}}^{\text{MC}}$ refer to the total data and non-QCD MC yields in the QCD CR.

In practice, this is complicated by the fact that a non-negligible amount of \ttbar signal is present in the QCD CR, whose cross section, as the parameter of interest in the measurement, is not known \textit{a priori}. %A modified method to correct for this is given in Appendix ??.
To circumvent this problem, a modified method is introduced, which is agnostic about the prediction for the \ttbar cross section. 
One sets for the SR

\begin{equation}
\label{eq:ttxs:abcd_modified1}
    N^{\mathrm{data}}_{\mathrm{SR}} = N^{\ttbar}_{\mathrm{SR}}+ N^{\mathrm{MC,BG}}_{\mathrm{SR}} + N^{\mathrm{QCD}}_{\mathrm{SR}}
\end{equation}

\noindent and similarly for the QCD CR

\begin{equation}
\label{eq:ttxs:abcd_modified2}
    N^{\mathrm{data}}_{\mathrm{CR}} = N^{\ttbar}_{\mathrm{CR}} + N^{\mathrm{MC,BG}}_{\mathrm{CR}} + N^{\mathrm{QCD}}_{\mathrm{CR}},
\end{equation}

\noindent where $N^{\mathrm{data}}$ is the total data yield, $N^{\ttbar}$ is the \ttbar signal contribution, $N^{\mathrm{MC,BG}}$ is the contribution of non-QCD backgrounds as predicted by MC, and $N^{\mathrm{QCD}}$ is the QCD contribution. It is assumed that the ratio $f_{\mathrm{sig}}$ of signal events in the SR and QCD CR (but not necessarily the normalization) is correctly predicted by MC:

\begin{equation}
\label{eq:ttxs:abcd_modified3}
    f_{\mathrm{sig}} := \frac{ N^{\ttbar}_{\mathrm{CR}} }{ N^{\ttbar}_{\mathrm{SR}} } = \frac{ N^{\mathrm{\ttbar,MC}}_{\mathrm{CR}} }{ N^{\mathrm{\ttbar,MC}}_{\mathrm{SR}} }
\end{equation}

Furthermore, one sets similar to \cref{eq:ttxs:abcd_naive}

\begin{equation}
\label{eq:ttxs:abcd_modified4}
    N^{\mathrm{QCD}}_{\mathrm{SR}} = N^{\mathrm{QCD}}_{\mathrm{CR}} \times f_{\mathrm{fake}}
\end{equation}

\noindent where $f_{\mathrm{fake}}$ is still given by \cref{eq:ttxs:abcd_fakerate}, which is unaffected since the \ttbar signal contamination in the 1-jet CR is negligible.

Combining all these equations, one can first replace $N^{\ttbar}_{\mathrm{CR}}$ in \cref{eq:ttxs:abcd_modified2} by $f_{\mathrm{sig}} N^{\ttbar}_{\mathrm{SR}}$ according to \cref{eq:ttxs:abcd_modified3}, then eliminate $N^{\ttbar}_{\mathrm{SR}}$ in favor of $N^{\mathrm{data}}_{\mathrm{SR}}$, i.e. the total data yield in the SR, and get

\begin{equation}
    N^{\mathrm{QCD}}_{\mathrm{SR}} = f_{\mathrm{fake}} \, \left( N^{\mathrm{data}}_{\mathrm{CR}} - N^{\mathrm{MC,BG}}_{\mathrm{CR}} - f_{\mathrm{sig}} \left( N^{\mathrm{data}}_{\mathrm{SR}} - N^{\mathrm{MC,BG}}_{\mathrm{SR}} - N^{\mathrm{QCD}}_{\mathrm{SR}} \right) \right) .
\end{equation}

Solving this equation for $N^{\mathrm{QCD}}_{\mathrm{SR}}$ finally yields the corrected QCD contribution in the SR:

\begin{equation}
\label{eq:ttxs:abcd_modified}
    N^{\mathrm{QCD}}_{\mathrm{SR}} = \left( N^{\mathrm{data}}_{\mathrm{CR}} - N^{\mathrm{MC,BG}}_{\mathrm{CR}} - f_{\mathrm{sig}} ( N^{\mathrm{data}}_{\mathrm{SR}} - N^{\mathrm{MC,BG}}_{\mathrm{SR}} )\right)
    \times \frac{ f_{\mathrm{fake}} } {1 - f_{\mathrm{sig}} f_{\mathrm{fake}}}
\end{equation}

The resulting QCD distributions from this method are further treated in the same way as the MC backgrounds, and can be seen together with them in \cref{fig:ttxs:control_em,fig:ttxs:control_eemm,fig:ttxs:control_ljets}.
  

\paragraph{Z+jets background}

In contrast to the QCD background, the Z+jets background is generally well-described by MC simulation. 
However, in the considered phase space, the requirement of at least one reconstructed b jet can introduce modeling challenges, as b quarks are treated as massless at the matrix-element level. This approximation may lead to inaccuracies in the predicted kinematic properties of b quarks compared to those observed in data.
%However, in the phase space used in the analysis, there can be problems related to the requirement of at least one reconstructed b jet. In Z+jets events, b quarks can in principle be produced as real emissions through higher-order corrections in QCD. However, in the LO simulation used here this is done by the parton shower, which might lead to poor modeling of the b quark properties compared to data. This in turn could influence the acceptance of Z+jets events, leading to an incorrect normalization in events with one or more b tags.

Here, a data-driven normalization is derived for Z+jets events with one or two b tags in the dilepton channels, following the method of \citere{CMS:EXO-22-014-PAS}. This is important especially in the same-flavor channels, where Z+jets is a dominant background.

The normalization is derived using a CR in which the cut on \mll is inverted, i.e. in events with $| \mll - m_Z | < 15$ GeV (``inside the Z window''), which are strongly enriched in Z+jets contributions. It is assumed that the Z+jets contribution in the \emu channel (which stems mostly from $\mathrm{Z} \rightarrow \tau \tau$ events) is negligible compared to the \ee and \mumu channels, and that all other backgrounds (including \ttbar) are approximately equal in the three dilepton channels up to combinatorics, in the sense that their differences are small compared to the Z+jets event yield. Then, said Z+jets yield in the Z window in the same-flavor channels can be estimated directly from data by subtracting the \emu channel -- and with it, the other backgrounds - from the \ee and \mumu channels. This results in

\begin{equation}
\label{eq:ttxs:zjets_yield}
    N_{\mathrm{ee / \mu\mu}}^{\mathrm{Z+jets}} = N_{\mathrm{ee / \mu\mu,\,in}}^{\mathrm{data}} - \frac{1}{2} N_{\mathrm{\emu,\,in}}^{\mathrm{data}} \, k_{\mathrm{ee / \mu\mu,\,in}}
\end{equation}

\noindent where $N_{\mathrm{\ell \ell},\,in}^{\mathrm{data}}$ refers to the number of observed events inside the Z window for the respective channel, and $k_{\mathrm{ee}} = k_{\mathrm{\mu\mu}}^{-1} = \sqrt{N_{\mathrm{ee,\,in}}^{\mathrm{data}} / N_{\mathrm{\mu\mu,\,in}}^{\mathrm{data}}}$ is a efficiency factor to correct for the different acceptance of electrons and muons.

To estimate the Z+jets background in the SR, the ratio $\Rinout = N_{\mathrm{in}}^{\mathrm{Z+jets}} / N_{\mathrm{out}}^{\mathrm{Z+jets}}$, defined as the number of Z+jets events inside and outside the Z mass window, needs to be determined. While this ratio could be taken directly from simulation (as done in \citeres{CMS:EXO-16-049,CMS:HIG-17-027}), it may be inaccurately modeled in MC. To reduce potential bias, a more conservative strategy is adopted. A second CR with zero b-tagged jets, which is not used in the main measurement for the same-flavor channels, is introduced to estimate the ratio under looser assumptions:

%To translate this yield from the CR to the SR, the ratio $\Rinout = N_{\mathrm{in}}^{\mathrm{Z+jets}} / N_{\mathrm{out}}^{\mathrm{Z+jets}}$ (referring to \textit{inside} and \textit{outside} of the Z window) of event numbers between those two regions has to be estimated. This could in principle be done by directly using the MC simulation (as done in e.g. \citeres{CMS:EXO-16-049,CMS:HIG-17-027}). However, since this ratio might by itself be mismodeled in MC, a more cautious approach is taken here. A second CR is defined from events with 0 b tags (which are not considered in the main measurement in the same-flavor channels) and used to construct to construct a more loose assumption:

\begin{equation}
    \frac{  \Rinout^{\text{data}} ( \geq \text{1 b tag} ) } { \Rinout^{\text{MC}} ( \geq \text{1 b tag} ) } = \frac{  \Rinout^{\text{data}} ( \text{0 b tags} ) } { \Rinout^{\text{MC}} ( \text{0 b tags} ) }
\end{equation}

This equation means that the \textit{ratio of ratios} $\Rinout( \geq \text{1 b tag} ) / \Rinout( \text{0 b tags} )$ is assumed to be well described by MC.
It can be solved for the Z+jets yield outside of the Z window in the same-flavor channels, yielding

\begin{equation}
\begin{split}
    N_{\mathrm{out}}^{\mathrm{Z+jets}} &= \frac {N_{\mathrm{in}}^{\mathrm{Z+jets}}} {\Rinout^{\text{data}} ( \geq \text{1 b tag} )}  \\
    &= \frac{  \Rinout^{\text{MC}} ( \text{0 b tags} ) } { \Rinout^{\text{data}} ( \text{0 b tags} ) } \, \frac {N_{\mathrm{in}}^{\mathrm{Z+jets}}} {\Rinout^{\text{MC}} ( \geq \text{1 b tag} )}
\end{split}
\end{equation}

\noindent where $N_{\mathrm{in}}^{\mathrm{Z+jets}}$ is given by \cref{eq:ttxs:zjets_yield}. In practice, this yield is quoted  as a scale factor compared to the nominal MC prediction. For the \emu channel (in which Z+jets is much less important), the scale factor is simply assumed to be the geometric mean of the \ee and \mumu scale factors.

\begin{table}[t]
    \begin{centering}
    \begin{tabular}{c|c|c}
    \ee & \emu & \mumu \tabularnewline
    \hline
    \hline
    $1.36 \pm 0.04$ & $1.32 \pm 0.03$ & $1.28 \pm 0.03$
    \end{tabular}
    \par\end{centering}
    \caption{\textbf{Z+jets scale factors.} Ratio of the Z+jets event yields estimated in data using the method described in \cref{sec:ttxs:datadriven} to the prediction by the MC simulation. Uncertainties are statistical only.}
    \label{tab:ttxs:dysf}
\end{table}

The final scale factors can be seen in \cref{tab:ttxs:dysf}.

\section{Control distributions}
\label{sec:ttxs:control}

The agreement between simulation and data in several control distributions is presented in \cref{fig:ttxs:control_em,fig:ttxs:control_eemm,fig:ttxs:control_ljets}. All corrections described in the previous section are applied in these figures. In addition, they are scaled by the b tagging efficiency scale factors obtained in the final likelihood fit (\cref{sec:ttxs:fitresults}) to better reflect the estimates for essential calibrations.

\begin{figure}[!hp]
\centering
\includegraphics[width=0.49\textwidth]{figures/ttxs/lep_pt_em.pdf}
\hfill
\includegraphics[width=0.49\textwidth]{figures/ttxs/lep_eta_em.pdf}
\includegraphics[width=0.49\textwidth]{figures/ttxs/1st_jet_pt_em.pdf}
\hfill
\includegraphics[width=0.49\textwidth]{figures/ttxs/mll_em.pdf}
\includegraphics[width=0.49\textwidth]{figures/ttxs/njet_em.pdf}
\hfill
\includegraphics[width=0.49\textwidth]{figures/ttxs/nbtag_em.pdf}
\caption{
    \textbf{Control distributions in the \emu channel.} Shown are (from top left to bottom right) the distributions of \pt of both leptons, \abseta of both leptons, \pt of the leading jet, the invariant lepton mass \mll, the number of jets and the number of b jets. All figures show both data (black dots) and different simulated background processes (colored bars). For the latter, all corrections described in \cref{sec:ttxs:corrections} as well as post-fit b tagging scale factors (\cref{sec:ttxs:fitresults}) are applied, and the shaded area covers both statistical and systematic uncertainties. \textit{Figure taken from \citere{CMS:TOP-22-012}}.
}
\label{fig:ttxs:control_em}
\end{figure}

\begin{figure}[!hp]
\centering
\includegraphics[width=0.49\textwidth]{figures/ttxs/lep_pt_eemm.pdf}
\hfill
\includegraphics[width=0.49\textwidth]{figures/ttxs/lep_eta_eemm.pdf}
\includegraphics[width=0.49\textwidth]{figures/ttxs/1st_jet_pt_eemm.pdf}
\hfill
\includegraphics[width=0.49\textwidth]{figures/ttxs/mll_eemm.pdf}
\includegraphics[width=0.49\textwidth]{figures/ttxs/njet_eemm.pdf}
\hfill
\includegraphics[width=0.49\textwidth]{figures/ttxs/nbtag_eemm.pdf}
\caption{
    \textbf{Control distributions in the \ee and \mumu channels.} The distributions are shown in the same manner as in Fig.~\ref{fig:ttxs:control_em}. \textit{Figure taken from \citere{CMS:TOP-22-012}}.
}
\label{fig:ttxs:control_eemm}
\end{figure}

\begin{figure}[!hp]
\centering
\includegraphics[width=0.49\textwidth]{figures/ttxs/lep_pt_lj.pdf}
\hfill
\includegraphics[width=0.49\textwidth]{figures/ttxs/lep_eta_lj.pdf}
\includegraphics[width=0.49\textwidth]{figures/ttxs/1st_jet_pt_lj.pdf}
\hfill
\includegraphics[width=0.49\textwidth]{figures/ttxs/1st_jet_eta_lj.pdf}
\includegraphics[width=0.49\textwidth]{figures/ttxs/njet_lj.pdf}
\hfill
\includegraphics[width=0.49\textwidth]{figures/ttxs/nbtag_lj.pdf}
\caption{
   \textbf{Control distributions in the \ljets channels.} The distributions are shown in the same manner as in Fig.~\ref{fig:ttxs:control_em}, except for the center-right figure, which here shows \abseta of the leading jet. \textit{Figure taken from \citere{CMS:TOP-22-012}}.
}
\label{fig:ttxs:control_ljets}
\end{figure}

Good agreement between data and simulation within the full uncertainties is seen in all distributions.

%\section{Likelihood fit}
%\label{sec:ttxs:fit}


\section{Systematic uncertainties}
\label{sec:ttxs:systematics}

% required: general shape uncertainties, special treatment of btag uncertainty, special treatment of lepton uncs in free floating case, XS uncs, lumi unc.

In order to translate the distribution of observed and expected events into a result for the inclusive \ttbar cross section while taking into account all relevant sources of systematic uncertainties, a binned profile maximum likelihood fit as described in \cref{sec:methods:stat} is performed using the tool \texttt{combine}~\cite{CMS:CAT-23-001}.
The parameter of interest (POI) used for this fit is the signal strength $r = \sigmatt / \sigmatt^{\text{pred}}$, i.e. the inclusive \ttbar cross section normalized to its theoretical prediction. A linear signal model is used as defined in \cref{eq:methods:linearsignal}, and the \ttbar cross section is extracted using its maximum likelihood estimate and uncertainty.

This section describes the considered systematic uncertainties, which can be divided into experimental uncertainties, arising from incomplete knowledge of the details of the detector and resulting differences between data and simulation, and theoretical uncertainties, which concern imperfect modeling of the underlying physical processes in the different event generators. %All of them will be described in this section.

All systematic uncertainties are included in the fit as nuisance parameters (NPs) as discussed in \cref{sec:methods:stat}. In practice, NPs which encode shape effects on the considered observables are implemented using \textit{template morphing}, i.e. a smooth polynomial interpolation between the nominal shape and the shapes encoding the variations by $\pm 1$ standard deviations. NPs that encode only normalization effects are instead implemented as simple log-normal uncertainties. Both definitions can be found in detail in \citere{CMS:CAT-23-001}.

Special attention is given in this section to some experimental uncertainties which are important to this measurement. This includes the luminosity, which is the dominating uncertainty, as well as 
%the lepton identification and 
the b tagging uncertainties due to the special way they are treated in the fit.

\paragraph{Luminosity uncertainty}

In order to translate event yields into a result on any cross section, the total integrated luminosity is required as a calibration constant. Any experimental error on the luminosity will be directly transferred to the total error on the measurement, and thus minimizing the luminosity uncertainty is crucial for any cross section measurement.

For the data set used in this analysis, the total integrated luminosity was measured by the CMS Collaboration with an estimated uncertainty of 2.3\%. Of this number, 2.1\% is due to the calibration of the integrated luminosity, using the methods presented in \citere{CMS:LUM-17-003}.%The largest contribution comes from the so-called \textit{factorization bias}, which arises in the van der Meer method from the assumption that the transverse luminous area factorizes in the $x$ and $y$ coordinates, and from residual beam position deviations.

The agreement in the absolute scale is checked by comparing different independently calibrated luminosity measurements. The integrated luminosity measured with the hadronic forward calorimeter and the silicon pixel detector is found to agree at a level of better than 0.8\%.
Accounting for residual differences in time stability and linearity between the luminosity detectors results in a total uncertainty of 2.3\%. This preliminary estimate of the integrated luminosity at the time of publication was further validated using the yield of reconstructed Z bosons decaying into muon pairs~\cite{CMS:DP-2023-003}. After correcting for efficiencies and normalizing to the fiducial cross section predicted at NNLO with next-to-NNLL corrections, good agreement was observed.\footnote{Since publication of this result, a more precise luminosity measurement for 2022 data has become available in \citere{CMS:LUM-22-001-PAS}.}
%Taking additional contributions due to residual differences in the time-stability and linearity between the luminosity detectors into account leads to the full figure of 2.3\%.
%The preliminary estimate of the integrated luminosity at the time of publication was further cross-checked by using the yield of reconstructed Z bosons decaying into pairs of muons~\cite{CMS:DP-2023-003}, corrected for efficiencies and normalized to the fiducial cross section prediction calculated at NNLO with next-to-NNLL corrections applied, which also showed good agreement.\footnote{Since publication of this result, a more precise luminosity measurement for 2022 data has become available in \citere{CMS:LUM-22-001-PAS}.}

In contrast to all other uncertainties described below, the uncertainty in the integrated luminosity is not directly included in the likelihood fit, but rather treated as an external uncertainty and added in quadrature afterwards, since it is expected to factorize completely from all other uncertainties.
The impact of varying the normalization of the backgrounds estimated from simulation by the integrated luminosity uncertainty was found to be negligible.
% TODO.

\paragraph{b tagging uncertainty}

As mentioned in \cref{sec:ttxs:scalefactors}, the efficiency for correctly identifying a jet originating from a b quark (b tagging) is expected to be different in data and simulation. At the time of this measurement, directly after the start of Run 3, no general-purpose b tagging studies had been available. Thus, the approach adopted here is to consider the b tagging efficiency in data to be completely unknown and measure it concurrently with the cross section in the likelihood fit.

For this purpose, the probability for an event with $n_{\mathrm{jet}}$ selected jets to have $n_{\mathrm{b tag}}$ correctly identified b jets, depending on the assumed b tagging efficiency $\epsilon_b$, is assumed to be a multinomial of the form

\begin{equation}
\label{eq:ttxs:btags}
    P (n_{\mathrm{b tag}} | n_{\mathrm{jet}}) \propto \epsilon_b^{n_{\mathrm{b tag}}} (1 - \epsilon_b)^{ n_{\text{no tag}}}
\end{equation}

Here, $n_{\text{no tag}}$ is the number of true b jets in the event which fall into the acceptance of the selection, but fail to be tagged by \textsc{DeepJet}. It is estimated from MC simulation. %Note that this might be lower then the number of b jets expected from the physical process (e.g. 2 for \ttbar). 

By taking the ratio of eq. \ref{eq:ttxs:btags} in data and simulation, one can derive a per-event weight which corrects the number of b tags in MC:
%. From this, a shape template depending on $\epsilon_b$ in data is constructed and included in the fit as a nuisance parameter. It can be seen in Fig. ??.

\begin{equation}
%\begin{split}
\label{eq:ttxs:btagsf}
    w_b = \frac
    {(\epsilon_b^{\mathrm{data}})^{n_{\mathrm{b tag}}} (1 - \epsilon_b^{\mathrm{data}})^{n_{\text{no tag}}}}
    {(\epsilon_b^{\mathrm{MC}})^{n_{\mathrm{b tag}}} (1 - \epsilon_b^{\mathrm{MC}})^{n_{\text{no tag}}}} 
    = (f_b)^{n_{\mathrm{b tag}}} \, \left( \frac{1 - f_b \, \epsilon_b^{\mathrm{MC}}}{1 - \epsilon_b^{\mathrm{MC}}} \right)^{n_{\text{no tag}}}
%\end{split}
\end{equation}

Here, $f_b = \epsilon_b^{\mathrm{data}}/\epsilon_b^{\mathrm{MC}}$ is the unknown b tagging scale factor. It is left freely floating in the likelihood fit. This is technically implemented by producing shape templates from MC with $f_b$ varied up and down by a fixed value and interpolating in between. This shape template can be seen in \cref{fig:ttxs:btagsf}, where it is evident that the categorization in the number of b tags gives significant constraining power for $f_b$. In the 1b categories, the shape with respect to the number of jets deviates significantly from a flat variation proportional to $f_b$ as naively expected. This is because of out-of-acceptance jets, corresponding to the second factor in \cref{eq:ttxs:btagsf}.

\begin{figure}[ht]
    \centering
    \includegraphics[width=0.99\textwidth]{figures/ttxs/scalefactors/btagsf.pdf}
    \caption{
       \textbf{b tagging scale factor variation.} The effect of varying the b tagging scale factor $f_b$ in \ttbar MC by an arbitrary value of $\pm 0.1$, shown for the number of jets in the 11 fit categories.
    }
    \label{fig:ttxs:btagsf}
\end{figure}

%Note that any dependence of $\epsilon_b$ on the jet kinematics factorizes out as long as said dependency is the same in data and MC. Possible kinematic dependencies of the ratio $f_b$ are neglected; since no kinematic information is used in the fit, this is deemed acceptable.
Note that, since $f_b$ is taken to be a single number, this method only corrects the overall b jet efficiency and does not consider any dependence of $\epsilon_b$ on jet kinematics. Because this measurement uses the same jet quality requirements (particularly the same \pt cuts) in all channels, and assuming that the b jet \pt and $\eta$ spectra in the different channels are roughly similar, any kinematic dependence is effectively integrated out in the overall efficiency scale factor $f_b$. %The fact that the spectrum itself is not corrected is not considered an issue here since the likelihood fit does not use kinematic information directly.
The lack of corrections to the spectrum is not considered problematic in this context, as the likelihood fit does not rely directly on kinematic information.

\paragraph{Lepton identification uncertainty}

% TODO: the structure is kinda shit here - scale factors + uncertainty + cross check. maybe try to merge some of these?

The uncertainty assumed on the lepton identification scale factors comes from two different sources: First, an inherent uncertainty originating in the tag-and-probe method (as described in \cref{sec:ttxs:scalefactors}) is considered. It consists of statistical uncertainties from both data and simulation, a systematic uncertainty derived from a comparison with a different Z+jets simulation sample produced at NLO in QCD, and another systematic uncertainty due to the choice of fitting function. Together, they make up for an uncertainty of $\sim 0.8\% \, (0.5\%)$ on the electron (muon) scale factors in the bulk of the phase space, and can rise up towards $\sim 5\%$ for high lepton \pt.

Secondly, it is taken into account that the scale factor between data and simulation might be slightly different in the Z+jets selection used for the T\&P method and the \ttbar selection used for the measurement of the cross section. The most important reason for this is the requirement of (b tagged) jets in almost all considered categories, as well as the requirement for at least three jets in the lepton+jets channels. 
This effect has been studied at CMS in the past and the difference found to be less then 0.5\% for muons and 1.0\% for electrons. Taking a conservative approach, these values are used as an additional component in the respective uncertainties.

In the first, preliminary version of this measurement~\cite{CMS:TOP-22-012-PAS}, the dedicated lepton efficiency scale factors as measured with the T\&P method were not yet available, and a different approach was taken. Similar to the b tagging efficiency, the lepton efficiency scale factors were kept freely floating the likelihood fit. Due to the different dependency on the lepton efficiencies in the different lepton flavor channels, the fit was able to constrain the efficiencies to a precision of 2\%~\cite{CMS:TOP-22-012}. The resulting scale factors were later found to be in good agreement with those obtained from the T\&P method, serving as a valuable cross-check. However, this method ultimately led to less precision and was thus not used in the final result.
\todo{decide on removing as Alexander suggested}

\paragraph{Pileup uncertainty}

As described in \cref{sec:ttxs:scalefactors}, three different pileup-related variables are employed to reweight the simulation to the observed data, and the average of the three weights is used as the nominal value. This method is repeated using only one of the variables - the number of good reconstructed vertices $n_{\mathrm{PV}}$ - and the difference in expected yields treated as an uncertainty. 
This procedure was compared to the usual estimation of pileup-related uncertainties in CMS. There, the theoretical expectation for the number of interactions %is derived as a function of total inelastic proton-proton cross section, and the latter is than varied by its experimental uncertainty.
%It was found that the heuristic method used here leads to larger uncertainties, and can thus be considered more conservative.
is taken as the product of the instantaneous luminosity and the total inelastic cross proton-proton cross section of $\SI{69.2}{\milli\barn} \pm 4.6 \%$ at \sqrtsRII~\cite{CMS:LUM-17-003}. It was found that the heuristic method used here leads to
larger uncertainties than the one from the inelatic cross section, and can thus be considered more conservative.

\paragraph{Jet energy uncertainties}

Uncertainties in the jet energy calibration are split into 26 different sources concerning different experimental and theoretical effects, following the standard CMS procedure outlined in \citere{CMS:JME-13-004}. 17 of these sources are found to be non-negligible and included in the fit, while the others are indistinguishable from fluctuations due to limited MC statistics. These sources include, among others, uncertainties due to jet \pt resolution and jet flavor composition, statistical uncertainties in the derivations of the energy corrections, and residual differences between data and simulation.

\paragraph{Trigger uncertainties}

Since the trigger scale factors are derived using the tag-and-probe method in the same way as the lepton scale factors, similar uncertainties are applied, including the uncertainties of 0.5\% for muons and 1.0\% for electrons due to extrapolation between Z+jets and \ttbar topologies. The only difference is that in the dilepton channels the uncertainties need to be propagated according to \cref{eq:ttxs:triggersf}. This has the effect of greatly reducing the impact of the trigger uncertainties in those channels compared to the lepton ID uncertainties, since the nominal per-event trigger efficiency is already very close to one.

\paragraph{Matrix element scale uncertainties}

The theoretical predictions of both signal and background are calculated using matrix elements at either LO or NLO in perturbative QCD, matched to a parton shower. Since this effectively means truncating the perturbative expansion of the scattering amplitude at a given power in the strong coupling constant, the effect of higher-order terms is neglected in the calculation.

At the same time, the necessity of renormalization of divergent diagrams and factorization of non-perturbative contributions introduces non-physical parameters into the prediction in the form of the renormalization and factorization scales $\mu_R$ and $\mu_F$ (cf. \cref{sec:mc:me}). These parameters are usually set to typical energy scales of the considered process, and might also depend on the event kinematics (dynamic scales).

To estimate possible uncertainties due to these missing terms as well as due to the choice of scales, the scales $\mu_R$ and $\mu_F$ are varied separately by a factor of 2 up and down, and the resulting change in simulation is taken as an uncertainty in the form of shape templates \cite{Cacciari:2004}. %In order to not double-count uncertainties in the cross section prediction for the backgrounds (see below), but keep possible rate variations due to acceptance effects, the templates are normalized to the nominal cross section values before any selection cuts are applied. Different physical processes are considered to be uncorrelated since they are produced with different generators and at different orders. 
To avoid double-counting uncertainties in the background cross section predictions (see below) while still accounting for possible rate variations due to acceptance effects, the templates are normalized to the nominal cross section values before applying any selection cuts.

\paragraph{PDF uncertainties}

The PDFs used to evaluate the non-perturbative contribution of the proton-proton collision have systematic uncertainties attached. They are estimated by independently reweighting the simulation to 100 different replicas of the used NNPDF 3.1 PDF set and taking the envelope of the resulting changes, following the recommendations of the PDF4LHC working group \cite{Butterworth:2015oua}. Additionally, the effect of the choice of the strong coupling constant in the PDF is assessed using a similar reweighting, and attached as a separate nuisance parameter. Analogously to the matrix element uncertainties, the resulting variations are normalized before any selection cuts to keep acceptance and shape effects while not double-counting cross section changes.

\paragraph{Parton shower uncertainties}

The parton shower model used for the predictions is only accurate (at most) at LL and LC in QCD (cf. \cref{sec:mc:showering}) and thus requires appropriate uncertainties. For this purpose, the scales at which the strong coupling constant is evaluated are varied up and down by a factor 2 separately for initial and final state radiation and for different processes, and the resulting changes are propagated to the fit as shape templates.

\paragraph{ME/PS matching uncertainty}

For the simulation of the \ttbar signal, an additional uncertainty concerning the matching between matrix element simulation in \powheg and parton showering in \pythia is considered. This is done by varying the $h_{\mathrm{damp}}$ parameter in \powheg controlling the amount of radiation generated at matrix element level, following \citere{CMS:TOP-16-021}.

\paragraph{Top quark \pt uncertainty}

It has been shown in previous measurements of \ttbar differential cross sections that the \pt spectrum of the top quark is significantly softer in data than in the standard \powheg MC simulation~\cite{CMS:TOP-17-014,CMS:TOP-16-007,CMS:TOP-16-008}. This effect is propagated to the \pt spectra of the top decay products and can thus lead to misestimation of the acceptance due to lepton and b jet \pt requirements. Fixed-order predictions at NNLO in QCD and NLO in EW are known to largely alleviate the discrepancy~\cite{Czakon:2017wor}. Thus, a common strategy is to reweight the top quark \pt spectrum in MC simulation to the one extracted from such fixed-order predictions.

At the time of the measurement, fixed-order predictions at NNLO in QCD and NLO EW were available only for \sqrtsRII and could not be directly applied to the MC simulation at \sqrtsRIII. Instead, the simulation is left uncorrected for the nominal prediction, and a variation is constructed by calculating the ratio of the fixed-order prediction from \citere{Czakon:2017wor} and the \powheg MC simulation at \sqrtsRII, and applying it to the \powheg MC simulation at \sqrtsRIII. The difference between uncorrected prediction and the variation is assigned as an additional uncertainty, which is one-sided by construction.

\paragraph{Background cross section uncertainties}

For the cross sections of the different processes, log-normal rate uncertainties are assigned based on the process and order at which it was calculated. Separate 15\% uncertainties are used for the $t$-channel single-top and tW backgrounds since they are generated at NLO with a NNLO prediction for the cross section, while
for W+jets and Diboson, 30\% is used since these samples are only generated at LO. For Z+jets, this is reduced to 20\% due to the data-driven estimation of the normalization.
Additionally, for the fully data-driven QCD background, two separate nuisance parameters for the \ejets and \mujets channels are defined, covering a conservative uncertainty of 30\% each.

\paragraph{Background statistical uncertainties}

Finally, since the background in this measurement is estimated either using MC simulation or data-driven methods, an independent statistical uncertainty needs to be attached to each bin, reflecting the finite number of events it contains. This is done using the so-called \textit{Barlow--Beeston light} method~\cite{Barlow:1993dm}. For MC backgrounds, these uncertainties are minuscule. However, for the data-driven QCD background, they also contain the propagated statistical uncertainty due to the limited number of data events in the CRs, which is in general non-negligible.


\section{Fit results}
\label{sec:ttxs:fitresults}

\begin{figure}[!ht]
\centering
\includegraphics[width=0.99\textwidth]{figures/ttxs/prefithist.pdf}
\includegraphics[width=0.99\textwidth]{figures/ttxs/postfithist.pdf}

\caption{
   \textbf{Comparison of data and simulation before and after the fit.} The distribution of the number of jets in the different fit categories is shown for data and simulation before (top) and after the likelihood fit (bottom). The fit greatly improves the agreement and strongly constrains the background uncertainties. \textit{Figure taken from \citere{CMS:TOP-22-012}}.
}
\label{fig:ttxs:prepostfit}
\end{figure}

Performing the fit yields a \ttbar signal strength of $r = 0.959 \pm 0.025$, where the uncertainty includes statistical and all systematic contributions, except for the 2.1\% uncertainty on the luminosity. This corresponds to an inclusive \ttbar cross section of

\[
    \sigmatt = 881 \pm 23 \, \text{(stat+syst)} \pm 20 \, \text{(lumi)} \, \text{pb}.
\]

The result is in good agreement with the standard model prediction of $\sigmatt^{\text{pred}} = 924^{+32}_{-40} \, \text{pb}$.

Fig.~\ref{fig:ttxs:prepostfit} shows the agreement between data and simulation before and after the fit. It can be immediately seen that the fit greatly reduces the uncertainty on the prediction by constraining systematic uncertainties and simultaneously improves the agreement compared to the data. 

Of particular note here is the free-floating b tagging efficiency (compare sec.~\ref{sec:ttxs:systematics}), whose effect can be directly read off from the categorization in the number of b jets: Before the fit (Fig.~\ref{fig:ttxs:prepostfit} top), the event yield for two or more b jets is overestimated in the simulation, while the yield for zero b jets is underestimated. This suggests that the b tagging efficiency is slightly lower in the data than assumed in the simulation. Indeed, the fit confirms this: the b tagging scale factor between data and simulation in the phase space of this measurement is measured to be $f_b  = 0.980 \pm 0.009$. As a result, after the fit (Fig.~\ref{fig:ttxs:prepostfit} bottom), the event yields agree in all b jet categories.

\subsection{Statistical checks}

To better understand the sources of systematic uncertainty, as well as the contributions of the different measurement channels, the fit is repeated twice, restricted to either the dilepton or the \ljets channels. For both cases as well as the combination, the contribution of different groups of nuisance parameters is calculated by freezing the groups to their postfit values and repeating the fit, as explained in \cref{sec:methods:stat}. It should be noted that this procedure does not take into account correlations between the groups, and thus the sum in quadrature of the separate components will in general not add up to the total uncertainty.

The results can be found in \cref{tab:ttxs:systematics}, where it can be seen how the combination of channels helps to reduce the total uncertainty: in the dilepton channels, the dominating uncertainties are the lepton identification uncertainty, which enters twice compared to the \ljets channels, as well as the statistical uncertainty of the data due to the relatively low branching ratio. In the \ljets channels, b tagging, JES, and pileup uncertainties dominate, reflecting the less clean selection and increased importance of jets. When the channels are combined, the uncertainty contribution of these groups lies inbetween the two separate numbers, showing how the channel combination represents a tradeoff between the advantages and disadvantages of either channel.

\begin{table}[!t]
\centering\renewcommand\arraystretch{1.1}
\begin{tabular}{l|c c c}
    Source & Full measurement & dilepton only & \ljets only\\
    \hline
    \hline
    Lepton ID efficiencies & 1.6 & 2.2 & 1.0 \\
    Trigger efficiency & 0.3 & \makebox[0pt][r]{$<$}0.1 & 0.5 \\
    JES & 0.6 & 0.7 & 1.1 \\
    b tagging efficiency & 1.1 & 0.8 & 2.1 \\
    Pileup reweighting & 0.5 & 0.2 & 1.1 \\
    ME scale, \ttbar & 0.5 & 0.4 & 0.5 \\
    ME scale, backgrounds & 0.2 & 0.1 & 0.3 \\
    ME/PS matching & 0.1 & 0.4 & 0.7 \\
    PS scales & 0.3 & 0.5 & 0.4 \\
    PDF and \alphas & 0.3 & 0.4 & 0.4 \\
    Top quark \pt & 0.5 & 0.3 & 0.5 \\
    tW background & 0.7 & 1.0 & 0.4 \\
    $t$-channel single-t background & 0.4 & \makebox[0pt][r]{$<$}0.1 & 0.5 \\
    Z+jets background & 0.3 & 0.2 & \makebox[0pt][r]{$<$}0.1 \\
    W+jets background & \makebox[0pt][r]{$<$}0.1 & \makebox[0pt][r]{$<$}0.1 & 0.2 \\
    Diboson background & 0.6 & 0.6 & \makebox[0pt][r]{$<$}0.1 \\
    QCD multijet background & 0.3 & -- & 0.5 \\
    Statistical uncertainty & 0.5 & 1.2 & 0.5 \\ \hline
    Combined uncertainty & 2.6 & 3.4 & 3.3 \\ \hline
    Integrated luminosity & 2.3 & 2.3 & 2.3 \\
\end{tabular}
\caption{
    \textbf{Sources of systematic uncertainty.} The relative per-cent contribution of different groups of sources of systematic uncertainty for the full measurement as well as for restrictions to the dilepton and \ljets channels only. They are calculated according to \cref{sec:methods:stat} and do not take correlations between the different groups into account.
}
\label{tab:ttxs:systematics}
\end{table}

Furthermore, the nuisance parameter pulls, constraints and impacts, as defined in \cref{sec:methods:stat}, are shown in \cref{fig:ttxs:impacts} for the channel combination. One can see how the electron identification scale factors, which are the leading impact, are constrained by the combination of channels, while the same is not true of the muon identification scale factors due to their lower prefit uncertainty.

\begin{figure}[!ht]
    \centering
    \includegraphics[width=0.9\linewidth]{figures/ttxs/impacts_v2.pdf}
    \caption{\textbf{Nuisance parameter pulls, constraints and impacts.} The expected and observed values are shown as shaded bands and error bars, respectively. Nuisance parameters are sorted by their observed impact on the signal strength $r$. For the b tagging scale factor, for which no prefit uncertainty is defined, the post-fit uncertainty is shown instead of the pull. \textit{Figure taken from the supplementary material of \citere{CMS:TOP-22-012}}.}
    \label{fig:ttxs:impacts}
\end{figure}


%\subsection{Lepton efficiency check}

\subsection{Top quark mass dependence}
\label{sec:ttxs:topmass}

An additional source of uncertainty that has not been considered so far is the choice of top quark mass in the \ttbar MC simulation. It affects the selection efficiency indirectly via the \pt cuts on leptons and jets, with higher top quark mass values leading to harder spectra and thus to larger efficiencies.

Contrary to other uncertainty sources, the top quark mass is not profiled in the likelihood fit. Instead, the dependence of the extracted \ttbar cross section is explicitly quantified as a function of the top quark mass by shifting its value in simulation by $\pm \SI{3}{\GeV}$ from its default of $\mt = \SI{172.5}{\GeV}$.
The extraction of \sigmatt is then repeated and the dependence on \mt extracted through a simple linear fit. This strategy has been taken in previous CMS and ATLAS \ttbar cross section measurements~\cite{CMS:TOP-17-001,ATLAS:2020aln}, and thus facilitates comparison with previous results.

For an upwards shift of $\Delta \mt = \SI{1}{\GeV}$, the \ttbar cross section is found to shift downwards by \SI{8.5}{\pb}, and vice versa. If one takes the current experimental uncertainty of \SI{0.3}{\GeV}~\cite{PDG:2022pth} as an allowed range for \mt, this would lead to an additional uncertainty on \sigmatt of 0.3\%.
% todo: impacts and GOF
% lepton SF consistency check
% tt curve

\section{Summary and Outlook}
\label{sec:ttxs:summary}

\begin{figure}[!ht]
    \centering
    \includegraphics[width=0.8\linewidth]{figures/ttxs/tt_curve.pdf}
    \caption{\textbf{Summary of \sigmatt measurements.} An overview of inclusive \ttbar cross section measurements at CMS at different center-of-mass energies~\cite{CMS:TOP-11-007, CMS:TOP-14-018, CMS:TOP-12-006, CMS:TOP-13-004, CMS:TOP-17-001, CMS:TOP-18-005, CMS:TOP-20-001, CMS:TOP-20-004} as well as comparison to the SM prediction~\cite{Czakon:2013goa}. This measurement is displayed as the red dot. \textit{Figure taken from \citere{CMS:TOP-22-012}}.}
    \label{fig:ttxs:ttcurve}
\end{figure}

In this chapter, the inclusive \ttbar cross section is measured for the first time at a center-of-mass energy of \sqrtsRIII. Data corresponding to an integrated luminosity of \lumiRIII from the beginning of LHC Run~3 are analyzed. Despite this comparatively small amount of data, a total precision of ca. 3\% with respect to the inclusive cross section is achieved.

\cref{fig:ttxs:ttcurve} compares the result of this chapter to other inclusive \ttbar cross section measurements performed by CMS at other center-of-mass energies~\cite{CMS:TOP-17-001, CMS:TOP-11-007, CMS:TOP-14-018, CMS:TOP-12-006, CMS:TOP-13-004, CMS:TOP-18-005, CMS:TOP-20-001, CMS:TOP-20-004}, as well as to the SM prediction~\cite{Czakon:2013goa}. The precision is comparable to other measurements at $\sqrt{s} = 7$, $8$, and $\SI{13}{\TeV}$, some of them with significantly higher integrated luminosities. All results are in agreement with the SM.

This measurement was designed specifically for the earliest data of Run~3, in order to achieve high precision without relying on a full suite of calibrations being available. In particular, b tagging and lepton efficiencies can be constrained in situ using the combination of dilepton and \ljets channels as well as the categorization by number of b-tagged jets. No large inconsistencies for any of the considered physics objects were found. The measurement was made public in September of 2022 just two months after the start of Run~3 and constituted the first public physics result of LHC Run~3. At the time, it provided a valuable first proof that CMS data taken in Run~3 were of high quality and ready for physics.

The next step for this result would be to transfer the technique developed in this work to well-understood data and high integrated luminosities in order to achieve the highest precision possible for \sigmatt. Such a measurement will certainly be dominated by systematic uncertainties, most importantly the luminosity and the lepton identification efficiencies (as already partly the case here). 
The channel combination method developed here could potentially help reduce the latter uncertainty through an in situ constraint, while the former is independent of the analysis strategy and requires more precise luminosity measurements for improvement. In addition, a more detailed study of the individual sources of uncertainty will likely be necessary to assess whether some can be reduced through improved calibrations.
%The channel combination method developed here could potentially help reduce the latter of these through an in situ constraint, while the former is orthogonal to the analysis strategy and its reduction requires more precise luminosity measurements. It will likely also be necessary to study the different sources of uncertainty in more detail, and investigate whether some of them can be reduced through more careful calibrations.

Additionally, one could try to use such a high-precision \ttbar cross section measurement to indirectly measure the top quark mass, one of the fundamental parameters of the Standard Model, by comparing the measured value of \sigmatt to SM predictions for different top quark masses. For this purpose, it would be important to reduce the dependence on the top quark mass in simulation (c.f. \cref{sec:ttxs:topmass}), for example by reducing the \pt requirements on leptons and jets as much as experimentally feasible. 
All of this leaves multiple parts for future studies to tread, which will be exciting to follow in the coming years as larger parts of the Run~3 data set are analyzed at CMS.
%All of this is, however, not part of this thesis and material for future work.
% what about outlook to higher lumi values? easy to do...


\begin{appendix}
\thispagestyle{empty}
\bibliography{literature}
\bibliographystyle{unsrt}
\end{appendix}

\end{document}
