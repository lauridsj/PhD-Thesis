\chapter{Introduction}
\label{ch:intro}

% begin : purpose of HEP - understand fundamental secrets of nature
% want to find new physics - but where to look
% promising: the top quark. heaviest SM particle - large Yukawa coupling to Higgs, new Higgs sectors, bare quark
% top quark colored, heavy, unstable - challenging to model in SM
% understanding top quark crucial for sm & bsm
% this thesis: study different aspects of ttbar production
% done as part of CMS, LHC, yadda yadda

It has always been the goal of high-energy physics to decipher the fundamental rules of nature. The most recent triumph in this journey was undoubtedly the discovery of the Higgs boson at the ATLAS and CMS experiments of the Large Hadron Collider (LHC) in 2012~\cite{ATLAS:2012tfa,CMS:HIG-12-028,CMS:HIG-12-036}, thus completing the Standard Model of particle physics (SM). Since then, the SM has been measured with ever-growing precision in the hope of finding and characterizing possible deviations from its predictions, and searches for new physics beyond the Standard Model (BSM) have been performed in various final states with complementary signatures. So far, no significant discrepancy has been found.

One promising avenue to search for new physics is the top quark, which is the most massive fundamental particle known today.
Because of its high mass, it could be particularly connected to new phenomena.
For example, if could be crucial in the search for new Higgs bosons, as they might exist in extended Higgs sectors, if their couplings to SM particles are Yukawa-like, i.e. proportional to the particle mass.
%It is thus particularly relevant in the search for new particles with Yukawa-like couplings, i.e. couplings proportional to the particle mass, such as extended Higgs sectors.
In addition, the top quark is the only colored particle with a short enough lifetime so that it decays before it hadronizes, and therefore allows measurements of properties like spin and polarization that are significantly more difficult to access for all other quarks. 
%It is also of interest in the context of the SM: as the only colored particle that decays before hadronizing, it allows for measurements of spin properties that are unaccessible for all other quarks. 

\medskip

In this thesis, different aspects of top quark pair (\ttbar) production in proton-proton collisions at the LHC are studied at the Compact Muon Solenoid (CMS) experiment~\cite{CMS:2008xjf}, which is one of the two large general-purpose LHC experiments. 
As a first test of top quark physics at the world record center-of-mass energy of \sqrtsRIII, a measurement of the inclusive \ttbar production cross section is presented~\cite{CMS:TOP-22-012}.
%The first topic is a measurement of the inclusive \ttbar production cross section, performed at a center-of-mass energy of \sqrtsRIII~\cite{CMS:TOP-22-012}. 
This energy was reached by the LHC for the first time in 2022 at the start of Run~3, after three years of shutdown and technical upgrades. The measurement performed here uses only \SI{1.21}{\fbinv} of data, taken in July and August 2022. To achieve a precision on the \ttbar cross section comparable with previous LHC measurements, experimental corrections such as lepton and b quark jet identification efficiencies habe been estimated \textit{in situ}, i.e. as part of the cross section measurement. The result was the first public result of LHC Run~3. It showed the high quality of the then-fresh collision data and provided a first confirmation of the SM at the new energy frontier.

% Run 3 of LHC: startup after shutdown, different calibrations etc
% new COM energy, energy frontier
% Run 3 ttbar xs measurement: first result of run 3
% validate data after shutdown

Second, in measurements of top quark properties and for searches of new particles connected to the top quark, a reliable modeling of the \ttbar production process is indispensible.
For this purpose, the Monte Carlo (MC) event generator \bbfourl~\cite{Jezo:2016ujg} is studied in great detail. It computes the full $\mathrm{p p} \rightarrow \bbllnunu$ matrix element for the dilepton decay channel of \ttbar and thus accurately models off-shell effects as well as the interference between \ttbar and tW production.
%Second, the modeling of \ttbar production in off-shell regions of phase space as well as the interference between \ttbar and tW production at CMS is studied~\cite{CMS:NOTE-2023-015}. This is done using the Monte Carlo (MC) event generator \bbfourl~\cite{Jezo:2016ujg}, which computes the full $p p \rightarrow b \bar{b} \ell^+ \ell^- \nu_{\ell} \bar{\nu}_{\ell}$ matrix element for the dilepton decay channel of \ttbar, thus naturally including off-shell and interference effects. 
\bbfourl is validated for the first time in the CMS simulation setup, and compared to other MC generators for \ttbar production, preparing its use in future precision measurements at CMS~\cite{CMS:NOTE-2023-015}.

% study on off-shell ttbar / tttW interference modeling
% bb4l: full off-shell matrix element
% validated in CMS, compared to other generators
% input for future ttbar precision measurements

Finally, a search for new spin-0 resonances decaying to \ttbar is presented, utilizing the complete CMS Run~2 data set corresponding to an integrated luminosity of \SI{138}{\fbinv}~\cite{CMS:TOP-24-007,CMS:HIG-22-013}. The analysis focuses on the dileptonic decay channel. In addition to the invariant mass of the \ttbar pair \mtt, spin correlation observables derived from the kinematics of the leptons and top quarks are employed to probe the spin and \CP nature of potential new intermediate resonances. Notably, a statistically significant excess over SM predictions is observed at low \mtt, with features consistent with a pseudoscalar hypothesis.
The excess is found to be consistent with a pseudoscalar \ttbar (quasi-)bound state, which is expected to exist in the SM according to non-relativistic QCD calculations, though its modeling remains a challenge. This constitutes the first time such a \ttbar bound state is experimentally observed.

%Finally, a search for new spin-0 states decaying to \ttbar is performed, using the full CMS Run~2 data set with a luminosity of \SI{138}{\fbinv}~\cite{CMS:TOP-24-007,CMS:HIG-22-013}. The dilepton decay channel of \ttbar is considered, and besides the invariant \ttbar mass (\mtt), spin correlation observables constructed from the leptons and top quarks are used to distinguish different spin states of the \ttbar system, giving sensitivity to the \CP structure of possible new states. Excitingly, a statistically significant excess is observed in the data compared to the standard SM predictions, which is located at low \mtt values and prefers pseudoscalar spin states. The excess is interpreted to be consistent with a pseudoscalar \ttbar (quasi-)bound state, which is expected to exist in the SM according to non-relativistic QCD calculations, though its modeling remains a challenge. This constitutes the first time such a \ttbar bound state is experimentally observed.

Since the experimental analysis cannot exclude possible BSM contributions to the excess, other possible interpretations are investigated, such as the existence of additional scalar or pseudoscalar bosons as expected e.g. in a Two-Higgs-Doublet Model. 
%Alternatively, the results are interpreted in terms of generic pseudoscalar or scalar bosons, as expected e.g. in a Two-Higgs-Doublet Model. 
The interference of these new bosons decaying to \ttbar and SM \ttbar production is taken into account, leading to complex signatures in the \mtt spectrum. In addition to an interpretation of the same low-\mtt excess, exclusion regions are derived for the presence of either one or two such new bosons in terms of their couplings to the top quark. For this purpose, the analysis of the dilepton decay channel of \ttbar is further combined with a separate analysis of the lepton+jets decay channel.

As a third interpretation of the excess, heavy Axion-Like Particles (ALPs) decaying to \ttbar are considered for the first time~\cite{Jeppe:2024sxt}. 
For the case that no explicit couplings between ALPs and gluons exist, experimental constraints on ALP couplings to top quarks are derived.
%After explicitly translating the experimental results for generic pseudoscalars to ALPs in the case of no explicit gluon couplings
For the case that ALP couplings to both top quarks and gluons exist, a phenomenological study of the discovery prospects of such heavy ALPs, as well as of the possibility to to distinguish them from other pseudoscalars, is performed. These results are presented for the currently available luminosity as well as future projections.

%The general case including explicit ALP-gluon couplings is further studied in simulated events. Sensitivity estimates for heavy ALPs coupling to top quarks, and for the prospects to distinguish them from other pseudoscalars, are derived for the currently available luminosity as well as future projections.

\medskip

The thesis is organized as follows. In \cref{ch:theory,ch:mc,ch:methods}, the framework used for the analyses is outlined. In particular, \cref{ch:theory} describes aspects of the SM and BSM theories relevant for this work, \cref{ch:methods} describes the LHC, the CMS detector, and object reconstruction methods, \cref{ch:mc} discusses the machinery of Monte Carlo generators, and \cref{ch:stat} briefly describes statistical inference techniques. 

Following that, \cref{ch:ttxs,ch:bb4l,ch:ah,ch:alps} each discuss one of the experimental or phenomenological results that comprise this work: the inclusive \ttbar cross section measurement in \cref{ch:ttxs}, the study of off-shell \ttbar production and \tttW interference in \cref{ch:bb4l}, the search for spin-0 states in \ttbar, including its interpretation as a \ttbar bound state, in \cref{ch:ah}, and the investigation of ALPs decaying to \ttbar in \cref{ch:alps}. Finally, a short summary and outlook is given in \cref{ch:summary}.

\Cref{ch:ttxs,ch:bb4l} fully consist of work done as part of this thesis, while in \cref{ch:ah,ch:alps} the major contributions from this thesis are summarized at the beginning of the chapters. 

% search for new states in ttbar spectrum, Run 2, full lumi
% BSM: heavy higgs bosons
% top quark decays before hadronizing: access to spin information
% excess is observed!! wow
% close to trheshold, pseudoscalar
% interpreted as ttbar bound state, first observation
% difficult to probe, difficult to model in QCD
% also: interpretation in terms of generic BSM pseudoscalars, scalars
% separately: interpretation as ALP - phenomenology study

% outline of the thesis

% cite all publications