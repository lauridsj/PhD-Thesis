\chapter{Introduction}
\label{ch:intro}

% begin : purpose of HEP - understand fundamental secrets of nature
% want to find new physics - but where to look
% promising: the top quark. heaviest SM particle - large Yukawa coupling to Higgs, new Higgs sectors, bare quark
% top quark colored, heavy, unstable - challenging to model in SM
% understanding top quark crucial for sm & bsm
% this thesis: study different aspects of ttbar production
% done as part of CMS, LHC, yadda yadda

It has always been the goal of high-energy physics to decipher the fundamental rules of nature. The most recent triumph in this journey was undoubtably the discovery of the Higgs boson at the ATLAS and CMS experiments of the Large Hadron Collider (LHC) in 2012~\cite{ATLAS:2012tfa,CMS:HIG-12-028,CMS:HIG-12-036}, thus completing the Standard Model of particle physics (SM). Since then, the SM has been measured with ever-growing precision in the hope of finding and characterizing possible deviations from its predictions, and searches for new phyiscs beyond the Standard Model (BSM) have been performed in many possible phase spaces and for many SM extensions. So far, no significant discrepancy has been found.

One promising avenue for searches for new physics is the top quark, which is the most massive fundamental particle in the SM. It is thus relevant for possible new particles with Yukawa couplings, i.e. couplings proportional to the particle mass, such as extended Higgs sectors. It is also of interest within the context of the SM: as the only colored particle that decays before hadronizing, it allows for measurements of spin properties, and poses challenges for its precision modeling in quantum chromodynamics (QCD). 

\medskip

In this thesis, different aspects of top quark pair (\ttbar) production in proton-proton collisions at the LHC are studied. It has been performed as part of the Compact Muon Solenoid (CMS) experiment~\cite{CMS:2008xjf}, which is one of the two large general-purpose LHC experiments. The first topic is a measurement of the inclusive \ttbar production cross section, performed at a center-of-mass energy of \sqrtsRIII~\cite{CMS:TOP-22-012}. This world-record energy was reached by the LHC for the first time in 2022 at the start of LHC Run~3, after three years of shutdown and technical upgrades. The measurement performed here uses only \SI{1.21}{\fbinv} of data, taken directly at the start of Run~3 in July and August 2022, to achieve a precision on the \ttbar cross section comparable with previous LHC measurements. This was made possible by designing a strategy aimed at estimating needed experimental corrections directly as part of the measurement. In addition to confirming the SM at a new energy, the result was the first public result of LHC Run~3 and showed the high quality of the then-fresh collision data.

% Run 3 of LHC: startup after shutdown, different calibrations etc
% new COM energy, energy frontier
% Run 3 ttbar xs measurement: first result of run 3
% validate data after shutdown

Second, the modeling of \ttbar production in off-shell regions of phase space as well as the interference between \ttbar and tW production at CMS is studied~\cite{CMS:NOTE-2023-015}. This is done using the Monte Carlo (MC) event generator \bbfourl~\cite{Jezo:2016ujg}, which computes the full $p p \rightarrow b \bar{b} \ell^+ \ell^- \nu_{\ell} \bar{\nu}_{\ell}$ matrix element for the dilepton decay channel of \ttbar, thus naturally including off-shell and interference effects. \bbfourl is validated for the first time in the CMS simulation setup, and compared to other MC generators for \ttbar production, preparing its use in future precision \ttbar measurements at CMS.

% study on off-shell ttbar / tttW interference modeling
% bb4l: full off-shell matrix element
% validated in CMS, compared to other generators
% input for future ttbar precision measurements

Finally, a search for new spin-0 states decaying to \ttbar is performed, using the full CMS Run~2 dataset with a luminosity of \SI{138}{\fbinv}~\cite{CMS:HIG-22-013-PAS,CMS:TOP-24-007}. The dilepton decay channel of \ttbar is considered, and besides the invariant \ttbar mass (\mtt), spin correlation observables constructed from the leptons and top quarks are used to distinguish different spin states of the \ttbar system, giving sensitivity to the \CP structure of possible new states. Excitingly, a statistically significant excess is observed in the data compared to the standard SM predictions, which is located at low \mtt values and prefers pseudoscalar spin states. The excess is interpreted to be consistent with a pseudoscalar \ttbar (quasi-)bound state, which is expected to exist in the SM according to non-relativistic QCD calculations, though its modeling remains a challenge. This constitutes the first time such a \ttbar bound state is experimentally observed.

Alternatively, the results are interpreted in terms of generic additional pseudoscalar or scalar bosons, as expected e.g. in a Two-Higgs-Doublet Model. The interference of these new bosons decaying to \ttbar and SM \ttbar production is taken into account, leading to complex signatures in the \mtt spectrum. In addition to an interpretation of the same low-\mtt excess, exclusion regions are derived for the presence of either one or two such new bosons in terms of their couplings to the top quark. For this purpose, the analysis of the dilepton decay channel of \ttbar is further combined with a separate analysis of the lepton+jets decay channel.

As a third interpretation, heavy Axion-Like Particles (ALPs) decaying to \ttbar are considered for the first time~\cite{Jeppe:2024sxt}. After explicitly translating the experimental results for generic pseudoscalars to ALPs in the limiting case of no explicit gluon couplings, the general case is studied in simulated events. Sensitivity estimates for heavy ALPs coupling to top quarks, and for the possibility to distinguish them from other pseudoscalars, are derived for the currently available luminosity as well as future projections.

\medskip

The thesis is organized as follows. In \cref{ch:theory,ch:mc,ch:methods}, the framework used for the analyses is outlined. In particular, \cref{ch:theory} describes aspects of the SM and BSM theories relevant for this work, \cref{ch:mc} briefly discusses the machinery of Monte Carlo generators as used for all relevant studies, and \cref{ch:methods} describes the LHC and the CMS detector as well as object reconstruction and statistical inference techniques. Following that, \cref{ch:ttxs,ch:bb4l,ch:ah,ch:alps} each discuss one of the experimental or phenomenological results that comprise this work: the inclusive \ttbar cross section measurement in \cref{ch:ttxs}, the study of off-shell \ttbar production and \tttW interference in \cref{ch:bb4l}, the search for spin-0 states in \ttbar, including its interpretation as a \ttbar bound state, in \cref{ch:ah}, and the investigation of ALPs decaying to \ttbar in \cref{ch:alps}. \Cref{ch:ttxs,ch:bb4l} fully consist of work done as part of this thesis, while in \cref{ch:ah,ch:alps} the major contributions from this thesis are summarized at the beginning of the chapters. Finally, a short summary and outlook is given in \cref{ch:summary}.

% search for new states in ttbar spectrum, Run 2, full lumi
% BSM: heavy higgs bosons
% top quark decays before hadronizing: access to spin information
% excess is observed!! wow
% close to trheshold, pseudoscalar
% interpreted as ttbar bound state, first observation
% difficult to probe, difficult to model in QCD
% also: interpretation in terms of generic BSM pseudoscalars, scalars
% separately: interpretation as ALP - phenomenology study

% outline of the thesis

% cite all publications