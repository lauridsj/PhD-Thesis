\chapter{Conclusions}
\label{ch:summary}

% studied several aspects of ttbar production at the LHC
% first measurement at run 3, only 1.21 fb, derived stuff in situ, precision, result, validation of new data

\section{Summary}

In this work, key measurements and searches for new physics are performed in  top quark pair (\ttbar) production with the CMS detector at the Large Hadron Collider (LHC). First, a measurement of the inclusive \ttbar production cross section at a world-record center-of-mass energy of \sqrtsRIII was performed, using \lumiRIII of early LHC Run~3 data. By combining the dilepton and \ljets decay channels of \ttbar for the first time and categorizing the events by their number of b tagged jets, the analysis is capable of constraining lepton and b tagging efficiencies directly \textit{in situ}. 

The result of $\sigmatt = 881 \pm 23 \, \text{(stat+syst)} \pm 20 \, \text{(lumi)} \, \text{pb}$ is compatible with the SM prediction within one standard deviation. The measurement became public only two months after the analyzed data set was recorded, constituting the first physics result of LHC Run~3. Despite the small luminosity, the precision of the result is comparable with previous \sigmatt measurements. At the time, it served as an important validation of the quality of CMS Run~3 data.

% bb4l: bbllnunu, interference, offshell rararara; validated at CMS for the first time, compared to other gens, found better description for mblminimax, shift in top mass, preparation for future precision measurements

\smallskip

Second, essential studies on the modeling of \ttbar production have been performed. In particular, off-shell \ttbar production as well as interference between \ttbar and tW production was studied in simulation using the Monte Carlo (MC) generator \bbfourl, which generates the full $\mathrm{pp} \rightarrow \bbllnunu$ amplitude at next-to-leading order (NLO) in quantum chromodynamics (QCD). %\bbfourl is expected to improve the description of \tttW interference and off-shell regions of the \ttbar phase space compared to previous \ttbar MC generators, which worked in the narrow-width approximation and thus required ad-hoc prescriptions for the interference treatment.
In this work, \bbfourl matched to the parton shower in \pythia is implemented and validated in the CMS software stack for the first time and compared to several other \ttbar MC generators. Good agreement between \bbfourl and unfolded ATLAS data is found for the variable \mblminimax, which is sensitive to the \tttW interference. Significant shifts in the reconstructed top mass line shape compared to other generators are observed. Additionally, a brief investigation of the matching procedure between \bbfourl and \pythia is presented. These studies represent the starting point for future precision \ttbar analyses in CMS using \bbfourl, such as measurements of the top mass and width.

% AH: search for spin-0 states in ttbar events, dilepton channel, full Run 2, invariant mass, spin corrs, top reco
% excess!!!! low mtt, pseudoscalar, interpretation as bound state, >5 sigma, cross section result for etat model, compatible order-of-magnitude as NRQCD, modeling of continuum biggest challenge
% also interpretation as generic A/H, e.g. 2HDM, set limits, combination with l+jets
% also: ALPs -> ttbar, interpretation for cG = 0, phenomenology, projected signifcances and limits for cG != 0, Run 3, HLLHC

\smallskip

Finally, a search for spin-0 states decaying to \ttbar in the dilepton channel has been presented. The search uses the full CMS Run~2 data set, corresponding to \lumiRII and \sqrtsRII, and employs the invariant \ttbar mass (\mtt) as well as two \ttbar spin correlation observables to gain sensitivity to the \CP structure of possible new intermediate states. An excess compared to the \ttbar continuum prediction is observed for low \mtt events, consistent with spin correlations as expected from a pseudoscalar state. This excess is interpreted as a pseudoscalar \ttbar bound state \etat, as predicted by several calculations in non-relativistic QCD (NRQCD). The production cross section of \etat is measured using a simplified \etat model, resulting in $\sigetat = 8.7 \pm 1.1  \, \si{\pb}$, which is of the same order of magnitude as NRQCD-based estimates. The uncertainty is dominated by its systematic component, in particular the challenging modeling of the \ttbar continuum. Several detailed cross-checks, such as bypassing the experimental \ttbar reconstruction as well as using different MC generators, are discussed, and all confirm the presence of the excess. The significance of the result exceeds five standard deviations. 
The excess has furthermore been confirmed by the ATLAS Collaboration in a preliminary result.

Because possible BSM contributions to the excess cannot be ruled out with the current experimental precision, 
the same search is further interpreted in terms of new, generic pseudoscalar (A) or scalar (H) particles coupling to top quarks, as expected e.g. in Two-Higgs Doublet models (2HDMs). The interference between the new particles and the SM is taken into account. %, leading to signals that are peak-dip structures in the \mtt spectrum. 
Besides an interpretation of the same excess at low \mtt, exclusion limits on the couplings to the top quark are presented for a large mass range in two scenarios, assuming the excess to be either fully described by A and H or fully by a \ttbar bound state. These limits are combined with a similar search in the \ljets decay channels of \ttbar. Together with a similar ATLAS result, they represent the most stringent limits to date for additional scalar or pseudoscalar bosons decaying to \ttbar. Furthermore, exclusion regions are also provided for the simultaneous presence of A and H, for the first time in the \ttbar final state.

For a third interpretation, Axion-Like Particles (ALPs) decaying to \ttbar are considered. In the case of vanishing tree-level couplings between ALPs and gluons \cG, the results for the generic pseudoscalar A are directly translatable, and experimental limits on the coupling between ALP and top quark are presented. The more generic case of $\cG \neq 0$ is studied using simulated events, and projected significances and exclusion limits on such ALPs decaying to \ttbar are derived for various phases of the LHC. It is found that at the high-luminosity LHC, ALPs and other pseudoscalars as e.g. in the 2HDM could be distinguishable based on their \mtt distribution. The resulting limits are expected to improve on limits from other final states in large areas of phase space.

\section{Outlook}

Branching out from the different aspects of this work, many directions of further study could be pursued. The most pressing one is certainly a further investigation of the excess at the \ttbar production threshold observed here. Since the ATLAS experiment has confirmed the excess in a preliminary result,
it would now be of great interest to scrutinize its origin, and to attempt to disentangle \ttbar bound state effects from possible BSM contributions.

\newpage

At the LHC, this will likely be challenging. For a direct measurement of the line shape, which could then be compared to predictions in the SM and in BSM models, the \mtt resolution would need to be increased by orders of magnitude. Applying modern machine learning techniques such as transformers or normalizing flows to the reconstruction of top quarks could present a first step in this direction. These improvements would also benefit top quark physics in general, e.g. for precision measurements of the top quark mass.

At the same time, more precise calculations of the \ttbar threshold are required. In particular, no fully differential Monte Carlo generator for the \ttbar threshold in proton-proton collisions is available at the time of writing, which required the use of toy models for a \ttbar bound state in this work. Replacing these by a rigorous first-principles calculation, which should simultaneously take into account bound state contributions and the \ttbar continuum, is indispensable to separate out any possible BSM effects. It will further be necessary to study the interplay and possible admixture between the bound state and BSM contributions.

Simultaneously, one should search for possible signals at the same resonance mass in other decay channels. For a \ttbar bound state, the decays to $\gamma\gamma$ and $Zh$ could possibly be observed, either already with the full Run~3 data or only at the high-luminosity LHC, depending on the uncertain branching fractions. Measurements of the ratio of these branching fractions and the branching fraction to \ttbar could then be compared to the predictions of specific BSM models, giving another handle for discriminating between \ttbar bound state and BSM. Further such handles could be provided by measurements of orthogonal observables in \ttbar events, e.g. color flow observables, for which machine learning-based reconstruction techniques are again crucial.

In the more distant future, a possible $e^+e^-$ collider running at a center-of-mass energy close to the \ttbar threshold would be invaluable to ascertain the nature of the excess. Due to the excellent energy resolution of lepton colliders, and the significantly cleaner environment due to reduced QCD effects, an energy scan of the \ttbar production cross section around the threshold would directly measure the line shape of the expected bound state peak. At the same time, the theory predictions of this line shape are available at great precision in QCD already now, and so a comparison of the energy scan results to the calculations could immediately exclude or confirm the presence of possible BSM contributions to the excess. A fit of the position and line shape of the peak could further be used for a precise determination of the top quark mass, width and Yukawa coupling, as well as the strong coupling \alphas.

%it would now be of great interest to determine the origin of the excess - in particular, whether it is purely the result of a SM bound state or whether it originates in BSM physics - though this will likely be challenging. Searches at the same invariant mass in other decay channels, in particular $\gamma\gamma$, as well as measurements of other kinematic distributions for low \mtt events could represent first steps towards this goal. It is also in general important to improve the experimental \ttbar reconstruction techniques e.g. with modern machine learning approaches, which would also greatly contribute to precision measurements of top quark properties. From the theoretical perspective, more precise calculations of the \ttbar threshold region are required, for which this work will hopefully serve as a motivation.

\smallskip

It is not every day that such an excess is observed in high energy physics. Regardless of its origin, its study marks the begin of a new chapter in top quark physics.

% some nice concluding remark: much further to be studied, characterization of the excess