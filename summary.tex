\chapter{Summary and Conclusions}
\label{ch:summary}

% studied several aspects of ttbar production at the LHC
% first measurement at run 3, only 1.21 fb, derived stuff in situ, precision, result, validation of new data

In this work, several aspects of top quark pair (\ttbar) production with the CMS detector at the Large Hadron Collider (LHC) were studied. First, a measurement of the inclusive \ttbar production cross section at a center-of-mass energy of \sqrtsRIII was performed, using \lumiRIII of early LHC Run~3 data. By combining the dilepton and \ljets decay channels of \ttbar for the first time and categorizing the events by their number of b-tagged jets, the analysis is capable of constraining lepton and b tagging efficiencies directly \textit{in situ}. 

The result is $\sigmatt = 882 \pm 23 \, \text{(stat+syst)} \pm 20 \, \text{(lumi)} \, \text{pb}$, compatible with the SM prediction within one standard deviation. The measurement became public only two months after data taking, constituting the first physics result of LHC Run~3, and despite the small considered luminosity is of comparable precision with previous \sigmatt measurements. At the time, it served as an important validation of the quality of CMS Run~3 data.

% bb4l: bbllnunu, interference, offshell rararara; validated at CMS for the first time, compared to other gens, found better description for mblminimax, shift in top mass, preparation for future precision measurements

\smallskip

Second, off-shell \ttbar production as well as interference between \ttbar and tW production was studied in simulation using the Monte Carlo (MC) generator \bbfourl, which generates the full $\mathrm{pp} \rightarrow \bbllnunu$ amplitude at next-to-leading order (NLO) in quantum chromodynamics (QCD). %\bbfourl is expected to improve the description of \tttW interference and off-shell regions of the \ttbar phase space compared to previous \ttbar MC generators, which worked in the narrow-width approximation and thus required ad-hoc prescriptions for the interference treatment.
In this work, \bbfourl matched to the parton shower in \pythia is implemented and validated in the CMS software stack for the first time and compared to several other \ttbar MC generators. Good agreement between \bbfourl and unfolded ATLAS data is found for the variable \mblminimax, which is sensitive to the \tttW interference, and significant shifts in the reconstructed top mass line shape compared to other generators are observed. Additionally, a brief investigation of the matching procedure between \bbfourl and \pythia is presented. These studies represent the starting point for future precision \ttbar analyses in CMS using \bbfourl, such as measurements of the top mass and width.

% AH: search for spin-0 states in ttbar events, dilepton channel, full Run 2, invariant mass, spin corrs, top reco
% excess!!!! low mtt, pseudoscalar, interpretation as bound state, >5 sigma, cross section result for etat model, compatible order-of-magnitude as NRQCD, modeling of continuum biggest challenge
% also interpretation as generic A/H, e.g. 2HDM, set limits, combination with l+jets
% also: ALPs -> ttbar, interpretation for cG = 0, phenomenology, projected signifcances and limits for cG != 0, Run 3, HLLHC

\smallskip

Finally, a search for spin-0 states decaying to \ttbar in the dilepton channel has been presented. The search uses the full CMS Run~2 data set, corresponding to \lumiRII and \sqrtsRII, and employs the invariant \ttbar mass (\mtt) as well as two \ttbar spin correlation observables to gain sensitivity to the \CP structure of possible new states. An excess compared to the \ttbar continuum prediction is observed for low \mtt events, consistent with spin correlations as expected from a pseudoscalar state. This excess is interpreted as a pseudoscalar \ttbar bound state \etat, as predicted by several calculations in non-relativistic QCD (NRQCD). The production cross section of \etat is measured using a simplified \etat model, resulting in $\sigetat = 8.7 \pm 1.1  \, \si{\pb}$, which is of the same order of magnitude as NRQCD-based estimates. The uncertainty is dominated by its systematic component, in particular the challenging modeling of the \ttbar continuum. Several detailed cross-checks, such as bypassing the experimental \ttbar reconstruction as well as using different MC generators, are discussed, and all confirm the presence of the excess. The significance of the result exceeds five standard deviations.%, giving the first observation of this process.

Alternatively, the same search is interpreted in terms of new, generic pseudoscalar (A) or scalar (H) particles coupling to top quarks, as expected e.g. in Two-Higgs Doublet models (2HDMs). The interference between the new particles and the SM is taken into account. %, leading to signals that are peak-dip structures in the \mtt spectrum. 
Besides an interpretation of the same excess at low \mtt, exclusion limits on the couplings to the top quark are presented in two scenarios, assuming the excess to be either fully described by A and H or fully by a \ttbar bound state. These limits are combined with a similar search in the \ljets decay channels of \ttbar, and exclusion regions are also provided for the simultaneous presence of A and H.

For a third interpretation, Axion-Like Particles (ALPs) decaying to \ttbar are considered. It is found that in the case of vanishing tree-level couplings between ALP and gluons \cG, the results for the generic pseudoscalar A are directly translatable, and experimental limits on the coupling between ALP and top quark are presented. The more generic case of $\cG \neq 0$ is studied using simulated events, and projected significances and exclusion limits on such ALPs decaying to \ttbar are derived for various phases of the LHC. It is found that at the high-luminosity LHC, ALPs and other pseudoscalars as e.g. in the 2HDM could be distinguishable based on their \mtt distribution. The resulting projected limits are expected to improve on limits from other final states in large areas of phase space.

\smallskip

Branching out from the different aspects of this work, many directions of further study could be pursued. The most pressing one is certainly a further investigation of the excess at the \ttbar production threshold observed here. Besides the outstanding confirmation or refutation from the ATLAS experiment, it would be of great interest to attempt to determine the origin of the excess - in particular, whether it is purely the result of a SM bound state or whether it originates in BSM physics - though this will likely be challenging. Searches at the same invariant mass in other decay channels, in particular $\gamma\gamma$, as well as measurements of other kinematic distributions for low \mtt events could represent first steps towards this goal. It is also in general important to improve the experimental \ttbar reconstruction techniques e.g. with modern machine learning approaches, which would also greatly contribute to precision measurements of top quark properties. From the theoretical side, more precise calculations of the \ttbar threshold region are required, for which this work will hopefully serve as a motivation.

It is not every day that such an excess is observed in high energy physics. One can only hope that, regardless of its origin, its study will produce many further results of great interest.

% some nice concluding remark: much further to be studied, characterization of the excess