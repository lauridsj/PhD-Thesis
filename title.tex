%!TEX root = main.tex
%\KOMAoptions{twoside=false}

%% Titlepage
\thispagestyle{empty}
\selectlanguage{ngerman}
\onehalfspacing

\begin{titlepage}
\begin{center}
  \vspace*{0cm}

  % Für kürzere Titel eventuell auf \huge setzen
  \LARGE %\huge
  \bfseries 
    Measurement of the inclusive \ttbar cross section and search for additional scalars in \ttbar final states at the CMS experiment

  \vspace{4cm}

  \normalfont
  \LARGE
    Dissertation\\

  \vspace{1cm}
  \normalfont
  \normalsize
    zur Erlangung des Doktorgrades \\
    an der Fakult{\"a}t f{\"u}r Mathematik, Informatik und Naturwissenschaften\\
    Fachbereich Physik\\
    der Universität Hamburg

 
  \vspace{1cm}
    von\\
  \bfseries 
  \large
   Laurids Jeppe

  \normalfont
  \large
  \vspace{3cm}
  Hamburg\\
  2025
\end{center}
\end{titlepage}


% \newpage
\thispagestyle{empty}
\onehalfspacing\normalsize\normalfont
  \vspace*{\fill}
\begin{tabular}{ll}
  Gutachter der Dissertation: & Prof. Dr. Christian Schwanenberger\\
  & Dr. Alexander Grohsjean\\
  \hspace{0pt}\\
  Zusammensetzung der Pr{\"u}fungskomission: & Prof. Dr. Christian Schwanenberger\\
  & Dr. Alexander Grohsjean\\
  & TBD\\
  & TBD\\
  & TBD\\
  \hspace{0pt}\\
  Vorsitzender der Pr{\"u}fungskomission: & TBD\\
  \hspace{0pt}\\
  Datum der Disputation: & TBD\\
  \hspace{0pt}\\
  Vorsitzender Fach-Promotionsausschuss Physik: & Prof. Dr. Wolfgang J. Parak\\
  \hspace{0pt}\\
  Leiter des Fachbereichs Physik: & Prof. Dr. Markus Drescher\\
  \hspace{0pt}\\
  Leiter der Fakult{\"a}t MIN: & Prof. Dr.-Ing. Norbert Ritter\\
\end{tabular}

\singlespacing
% reset settings basline will be reset in BA.tex

%\KOMAoptions{twoside=true}