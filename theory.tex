\chapter{Theoretical framework}
\label{ch:theory}

This chapter gives an outline of the theoretical concepts and models used in this thesis. It is split into two parts: First, the Standard Model of elementary particle physics is discussed, with a heavy emphasis on the top quark. Secondly, several hypothesized extensions of the Standard Model, relevant for the searches presented in ??, are briefly introduced and compared.

\section{Standard Model}

The Standard Model of elementary particle physics, often simply called the Standard Model or SM, is, at the time of writing, the most successful theory describing the fundamental particles making up our universe. It is the result of a steady progression of increasingly complex models, starting with the introduction of quantum mechanics in the early 20th century and ending - for now - with the discovery of the Higgs boson at the LHC in 2012. The model has been extensively tested at many different experiments, most importantly the large collider experiments like LEP, the Tevatron, and the LHC. So far, it has survived all these tests with excellence.

The SM is formulated as a relativistic quantum field theory (QFT). That is, its most fundamental objects are fields acting on four-dimensional spacetime, which, after a quantization procedure, yield physically observable particles as fundamental excitations. By the usual counting scheme, there exist seventeen different such fields, which can be classified into different groups. The first group consists of the twelve fermions, which have spin $\frac{1}{2}$ and make up all visible matter. They are further split into the leptons, consisting of three electrically charged leptons - electron, muon and tau lepton - and three corresponding electrically neutral neutrinos, as well as the quarks, of which there are six different flavors, called up, down, strange, charm, bottom, and top. The quarks have fractional electric charge, and in addition carry color charge, coupling to the strong interaction, as their defining property. Of note is that the fermions are also split into three generations, with each generation consisting of a charged lepton, a neutrino, and two quarks. The only fundamental differences between the particles of different generations are their masses, though the resulting physically observable properties, such as the lifetime, might be dramatically different.

The second group of particles are the bosons, which have integer spin. Here, the four gauge bosons with spin 1 act as the force carriers of the four fundamental interactions described by the SM: the photon, for the electromagnetic interaction; the W and Z bosons, for the weak interaction; and the gluon, for the strong interaction. The last and final particle is the Higgs boson, which has spin 0. Its most important role in the SM is to give mass to the fermions, as well as the W and Z bosons, through the so-called Higgs mechanism.


\subsection{Top quark}

\subsection{Higgs boson}

\section{The \pptt process}
\label{sec:theory:ttbar}

\subsection{Spin density matrix}

\subsection{Non-perturbative effects}

\section{Beyond the Standard Model}

\subsection{Extended Higgs sector models}
\label{sec:theory:hext}

\subsection{Axion-Like Particles}

\chapter{Monte Carlo event generation}

\section{Monte Carlo method}

\section{Matrix Element generators}

\section{Parton showers and matching}
