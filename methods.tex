\chapter{Experimental methods}
\label{ch:methods}

\section{The Large Hadron Collider}

\section{The CMS experiment}

\section{Data reconstruction}
\label{sec:methods:reco}

\section{Statistical interpretation}

In experimental particle physics, results are typically extracted by comparing detector-level predictions, for example obtained using MC simulation, to the observed data for suitably chosen observables. The measured data here are neccesarily afflicted by statistical uncertainties, both due to the inherent randomness of quantum mechanics and the probabilistic behaviour of the detector. They should thus be seen as a sample drawn from a random distribution, and in order to extract underlying parameters of any model, statistical methods are required.

In this work, all statistical interpretation is performed in the framework of \textit{binned profile maximum likelihood fits}. This method follows the Frequentist approach of considering physical properties that should be extracted to be fixed, if unknown, quantities, which enter the random distribution of the observed data as parameters. In order to estimate the desired properties, the observed datapoints are sorted into orthogonal bins according to one or more sensitive observables, and each bin is treated as an independent counting experiment where the observed number of events is given by a Poissonian distribution. 

Denoting the set of physical properties to be estimated (the parameters of interest or POIs) collectively as $\vec{\mu}$, the likelihood of $\vec{\mu}$ for bin $i$, given that $N_i$ events were observed, is

\begin{equation}
    L_i (\vec{\mu}, \vec{\theta}) = \mathrm{Pois} \left(N_i | n_i (\vec{\mu}, \vec{\theta}) \right)
\end{equation}

Here, $\mathrm{Pois}$ refers to the Poissonion distribution, and $n_i (\vec{\mu}, \vec{\theta})$ is the mean expected number of events in bin $i$ as predicted by the 